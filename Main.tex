% Options for packages loaded elsewhere
\PassOptionsToPackage{unicode}{hyperref}
\PassOptionsToPackage{hyphens}{url}
\PassOptionsToPackage{dvipsnames,svgnames,x11names}{xcolor}
%
\documentclass[
  letterpaper,
  DIV=11,
  numbers=noendperiod]{scrartcl}

\usepackage{amsmath,amssymb}
\usepackage{iftex}
\ifPDFTeX
  \usepackage[T1]{fontenc}
  \usepackage[utf8]{inputenc}
  \usepackage{textcomp} % provide euro and other symbols
\else % if luatex or xetex
  \usepackage{unicode-math}
  \defaultfontfeatures{Scale=MatchLowercase}
  \defaultfontfeatures[\rmfamily]{Ligatures=TeX,Scale=1}
\fi
\usepackage{lmodern}
\ifPDFTeX\else  
    % xetex/luatex font selection
  \setmainfont[]{Inter}
  \setsansfont[]{Inter}
\fi
% Use upquote if available, for straight quotes in verbatim environments
\IfFileExists{upquote.sty}{\usepackage{upquote}}{}
\IfFileExists{microtype.sty}{% use microtype if available
  \usepackage[]{microtype}
  \UseMicrotypeSet[protrusion]{basicmath} % disable protrusion for tt fonts
}{}
\makeatletter
\@ifundefined{KOMAClassName}{% if non-KOMA class
  \IfFileExists{parskip.sty}{%
    \usepackage{parskip}
  }{% else
    \setlength{\parindent}{0pt}
    \setlength{\parskip}{6pt plus 2pt minus 1pt}}
}{% if KOMA class
  \KOMAoptions{parskip=half}}
\makeatother
\usepackage{xcolor}
\setlength{\emergencystretch}{3em} % prevent overfull lines
\setcounter{secnumdepth}{5}
% Make \paragraph and \subparagraph free-standing
\ifx\paragraph\undefined\else
  \let\oldparagraph\paragraph
  \renewcommand{\paragraph}[1]{\oldparagraph{#1}\mbox{}}
\fi
\ifx\subparagraph\undefined\else
  \let\oldsubparagraph\subparagraph
  \renewcommand{\subparagraph}[1]{\oldsubparagraph{#1}\mbox{}}
\fi


\providecommand{\tightlist}{%
  \setlength{\itemsep}{0pt}\setlength{\parskip}{0pt}}\usepackage{longtable,booktabs,array}
\usepackage{calc} % for calculating minipage widths
% Correct order of tables after \paragraph or \subparagraph
\usepackage{etoolbox}
\makeatletter
\patchcmd\longtable{\par}{\if@noskipsec\mbox{}\fi\par}{}{}
\makeatother
% Allow footnotes in longtable head/foot
\IfFileExists{footnotehyper.sty}{\usepackage{footnotehyper}}{\usepackage{footnote}}
\makesavenoteenv{longtable}
\usepackage{graphicx}
\makeatletter
\def\maxwidth{\ifdim\Gin@nat@width>\linewidth\linewidth\else\Gin@nat@width\fi}
\def\maxheight{\ifdim\Gin@nat@height>\textheight\textheight\else\Gin@nat@height\fi}
\makeatother
% Scale images if necessary, so that they will not overflow the page
% margins by default, and it is still possible to overwrite the defaults
% using explicit options in \includegraphics[width, height, ...]{}
\setkeys{Gin}{width=\maxwidth,height=\maxheight,keepaspectratio}
% Set default figure placement to htbp
\makeatletter
\def\fps@figure{htbp}
\makeatother

\usepackage{amsmath, xparse}
\usepackage{fancyvrb, fvextra}
\usepackage{bm}
\usepackage{svg}
\usepackage{listings}
\usepackage{xifthen}
\DefineVerbatimEnvironment{Highlighting}{Verbatim}{breaklines,commandchars=\\\{\}}
\lstset{basicstyle=\ttfamily\footnotesize,breaklines=true}
\newcommand\rowop[1]{\scriptstyle\smash{\xrightarrow[\vphantom{#1}]{\mkern-4mu#1\mkern-4mu}}}
\DeclareDocumentCommand\converttorows%
{>{\SplitList{,}}m}%
{\ProcessList{#1}{\converttorow}}
\NewDocumentCommand{\converttorow}{m}
{\ifthenelse{\isempty{#1}}{}{\rowop{#1}}\\}

\DeclareDocumentCommand \rowops{m}
{\;
\begin{matrix}
\converttorows {#1}
\end{matrix}
\; }
\KOMAoption{captions}{tableheading}
\makeatletter
\makeatother
\makeatletter
\makeatother
\makeatletter
\@ifpackageloaded{caption}{}{\usepackage{caption}}
\AtBeginDocument{%
\ifdefined\contentsname
  \renewcommand*\contentsname{Table of contents}
\else
  \newcommand\contentsname{Table of contents}
\fi
\ifdefined\listfigurename
  \renewcommand*\listfigurename{List of Figures}
\else
  \newcommand\listfigurename{List of Figures}
\fi
\ifdefined\listtablename
  \renewcommand*\listtablename{List of Tables}
\else
  \newcommand\listtablename{List of Tables}
\fi
\ifdefined\figurename
  \renewcommand*\figurename{Figure}
\else
  \newcommand\figurename{Figure}
\fi
\ifdefined\tablename
  \renewcommand*\tablename{Table}
\else
  \newcommand\tablename{Table}
\fi
}
\@ifpackageloaded{float}{}{\usepackage{float}}
\floatstyle{ruled}
\@ifundefined{c@chapter}{\newfloat{codelisting}{h}{lop}}{\newfloat{codelisting}{h}{lop}[chapter]}
\floatname{codelisting}{Listing}
\newcommand*\listoflistings{\listof{codelisting}{List of Listings}}
\makeatother
\makeatletter
\@ifpackageloaded{caption}{}{\usepackage{caption}}
\@ifpackageloaded{subcaption}{}{\usepackage{subcaption}}
\makeatother
\makeatletter
\@ifpackageloaded{tcolorbox}{}{\usepackage[skins,breakable]{tcolorbox}}
\makeatother
\makeatletter
\@ifundefined{shadecolor}{\definecolor{shadecolor}{rgb}{.97, .97, .97}}
\makeatother
\makeatletter
\makeatother
\makeatletter
\makeatother
\ifLuaTeX
  \usepackage{selnolig}  % disable illegal ligatures
\fi
\IfFileExists{bookmark.sty}{\usepackage{bookmark}}{\usepackage{hyperref}}
\IfFileExists{xurl.sty}{\usepackage{xurl}}{} % add URL line breaks if available
\urlstyle{same} % disable monospaced font for URLs
\hypersetup{
  colorlinks=true,
  linkcolor={blue},
  filecolor={Maroon},
  citecolor={Blue},
  urlcolor={Blue},
  pdfcreator={LaTeX via pandoc}}

\author{}
\date{}

\begin{document}
\begin{titlepage}

    \newcommand{\HRule}{\rule{\linewidth}{0.5mm}}
    
    \center
    
    \vspace{10cm}

    \textsc{\LARGE Gwinnett School of Math, Science, and Technology }\\[0.3cm]
    
    \vspace{0.5cm}

    \HRule \\[0.4cm]
    { \huge \bfseries Multivariable Calculus Yearlong Notes}\\[0.03cm]
    \HRule \\[1.5cm]
    
    \begin{minipage}{0.4\textwidth}
    \begin{flushleft} \Large
    Anish Goyal \\1st Period
    \end{flushleft}
    \end{minipage}
    ~
    \begin{minipage}{0.4\textwidth}
    \begin{flushright} \Large
    Donny Thurston\\Educator
    \end{flushright}
    \end{minipage}\\[1cm]
    
    {\huge 2023-2024}\\[1cm]
    
    \includegraphics{img/logo.png}\\
    \vfill
    \end{titlepage}
\newpage

\ifdefined\Shaded\renewenvironment{Shaded}{\begin{tcolorbox}[frame hidden, interior hidden, enhanced, borderline west={3pt}{0pt}{shadecolor}, breakable, sharp corners, boxrule=0pt]}{\end{tcolorbox}}\fi

\renewcommand*\contentsname{Table of Contents}
{
\hypersetup{linkcolor=}
\setcounter{tocdepth}{4}
\tableofcontents
}
\newpage{}

\hypertarget{systems-of-linear-equations-and-matrices}{%
\section{Systems of Linear Equations and
Matrices}\label{systems-of-linear-equations-and-matrices}}

\hypertarget{matrix-operations}{%
\subsection{Matrix Operations}\label{matrix-operations}}

\begin{itemize}
\tightlist
\item
  Matrix operations are given as: rows x columns
\item
  Two matrices are equal \(\iff\) they have the same dimensions and
  values
\end{itemize}

\hypertarget{addition-subtraction}{%
\subsubsection{Addition \& Subtraction}\label{addition-subtraction}}

Two matrices can be added/subtracted \(\iff\) they have the same
dimensions. \includegraphics{img/addition-subtraction.png}

\hypertarget{scalar-multiplication}{%
\subsubsection{Scalar Multiplication}\label{scalar-multiplication}}

\begin{itemize}
\tightlist
\item
  Scalar multiplication is defined as multiplying each element of a
  matrix by a number \includegraphics{img/scalar.png}
\end{itemize}

\hypertarget{matrix-multiplication}{%
\subsubsection{Matrix Multiplication}\label{matrix-multiplication}}

\begin{itemize}
\tightlist
\item
  We can \textbf{only} multiply an (m x n) by (n x p) matrix.
\item
  The resulting matrix will be (m x p)
\end{itemize}

\hypertarget{examples}{%
\subsubsection{Examples}\label{examples}}

\begin{enumerate}
\def\labelenumi{\arabic{enumi}.}
\item
  \begin{align*}
  &\begin{bmatrix} 1 & 2 \\ 3 & 4\end{bmatrix}\begin{bmatrix} 1 & 2 \\ 3 & 4 \end{bmatrix} \\
  &= \begin{bmatrix} 1 \cdot 1 + 2 \cdot 3 & 1 \cdot 2 + 2 \cdot 4 \\ 3 \cdot 1 + 4 \cdot 3 & 3 \cdot 2 + 4 \cdot 4 \end{bmatrix} \\
  &= \begin{bmatrix} 7 & 10 \\ 15 & 22 \end{bmatrix}
  \end{align*}
\item
  \begin{align*}
  &\begin{bmatrix}2 & -3 \\ 5 & 0 \\ -2 & 4 \\ 1 & 2 \end{bmatrix}\begin{bmatrix}-1 \\ 3 \end{bmatrix} \\
  &=\begin{bmatrix} 2 \cdot (-1) + (-3) \cdot 3 \\ 5 \cdot (-1) + 0 \cdot 3 \\ -2 \cdot (-1) + 4 \cdot 3 \\ 1 \cdot (-1) + 2 \cdot 3 \end{bmatrix} \\
  &=\begin{bmatrix} -11 \\ -5 \\ 14 \\ 5 \end{bmatrix}
  \end{align*}
\item
  \begin{align*}
  &\begin{bmatrix}4 & 5 & -1 \end{bmatrix}\begin{bmatrix} 8 \\ 0 \\ 2\end{bmatrix} \\
  &= \begin{bmatrix} 4 \cdot 8 + 5 \cdot 0 + (-1) \cdot 2 \end{bmatrix} \\
  &= \begin{bmatrix} 30 \end{bmatrix}
  \end{align*}
\end{enumerate}

\hypertarget{transpose-of-a-matrix}{%
\subsection{Transpose of a Matrix}\label{transpose-of-a-matrix}}

The transpose of an (m x n) matrix is the (n x m) matrix where the rows
and columns are swapped.

If
\(B = \begin{bmatrix} 4 & 2 \\ -1 & 0 \\ 3 & 5 \end{bmatrix}, B^T = \begin{bmatrix} 4 & -1 & 3 \\ 2 & 0 & 5 \end{bmatrix}\)

\begin{align*}
B \cdot B^T &= \begin{bmatrix} 4 & 2 \\ -1 & 0 \\ 3 & 5 \end{bmatrix} \begin{bmatrix} 4 & -1 & 3 \\ 2 & 0 & 5 \end{bmatrix} \\ 
&= \begin{bmatrix} 4 \cdot 4 + 2 \cdot 2 & 4 \cdot (-1) + 2 \cdot 0 & 4 \cdot 3 + 2 \cdot 5 \\ (-1) \cdot 4 + 0 \cdot 2 & (-1) \cdot (-1) + 0 \cdot 0 & (-1) \cdot 3 + 0 \cdot 5 \\ 3 \cdot 4 + 5 \cdot 2 & 3 \cdot (-1) + 5 \cdot 0 & 3 \cdot 3 + 5 \cdot 5 \end{bmatrix} \\ 
&= \begin{bmatrix} 20 & -4 & 22 \\ -4 & 1 & -3 \\ 22 & -3 & 34\end{bmatrix}
\end{align*}

\begin{itemize}
\tightlist
\item
  The transpose of a matrix is \textbf{always} multiplicative with the
  original.
\item
  There is also a \textbf{main diagonal} that is the diagonal from the
  top left to the bottom right.
\item
  The \textbf{trace} of a square matrix \(A\) is equal to the sum of all
  the elements on the main diagonal: \(\mathrm{tr}(A)\)
\end{itemize}

\hypertarget{transpose-matrix-properties}{%
\subsubsection{Transpose Matrix
Properties}\label{transpose-matrix-properties}}

\begin{itemize}
\tightlist
\item
  \((A^T)^T = A\)
\item
  \((A + B)^T = A^T + B^T\)
\item
  \((A - B)^T = A^T - B^T\)
\item
  \((kA)^T = kA^T\)
\item
  \((AB)^T = B^T A^T\)
\end{itemize}



\end{document}
