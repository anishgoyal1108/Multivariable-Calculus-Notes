% Options for packages loaded elsewhere
\PassOptionsToPackage{unicode}{hyperref}
\PassOptionsToPackage{hyphens}{url}
\PassOptionsToPackage{dvipsnames,svgnames,x11names}{xcolor}
%
\documentclass[
  letterpaper,
  DIV=11,
  numbers=noendperiod]{scrartcl}

\usepackage{amsmath,amssymb}
\usepackage{iftex}
\ifPDFTeX
  \usepackage[T1]{fontenc}
  \usepackage[utf8]{inputenc}
  \usepackage{textcomp} % provide euro and other symbols
\else % if luatex or xetex
  \usepackage{unicode-math}
  \defaultfontfeatures{Scale=MatchLowercase}
  \defaultfontfeatures[\rmfamily]{Ligatures=TeX,Scale=1}
\fi
\usepackage{lmodern}
\ifPDFTeX\else  
    % xetex/luatex font selection
  \setmainfont[]{Inter}
  \setsansfont[]{Inter}
  \setmathfont[]{Fira Math}
\fi
% Use upquote if available, for straight quotes in verbatim environments
\IfFileExists{upquote.sty}{\usepackage{upquote}}{}
\IfFileExists{microtype.sty}{% use microtype if available
  \usepackage[]{microtype}
  \UseMicrotypeSet[protrusion]{basicmath} % disable protrusion for tt fonts
}{}
\makeatletter
\@ifundefined{KOMAClassName}{% if non-KOMA class
  \IfFileExists{parskip.sty}{%
    \usepackage{parskip}
  }{% else
    \setlength{\parindent}{0pt}
    \setlength{\parskip}{6pt plus 2pt minus 1pt}}
}{% if KOMA class
  \KOMAoptions{parskip=half}}
\makeatother
\usepackage{xcolor}
\ifLuaTeX
  \usepackage{luacolor}
  \usepackage[soul]{lua-ul}
\else
  \usepackage{soul}
  
\fi
\setlength{\emergencystretch}{3em} % prevent overfull lines
\setcounter{secnumdepth}{5}
% Make \paragraph and \subparagraph free-standing
\ifx\paragraph\undefined\else
  \let\oldparagraph\paragraph
  \renewcommand{\paragraph}[1]{\oldparagraph{#1}\mbox{}}
\fi
\ifx\subparagraph\undefined\else
  \let\oldsubparagraph\subparagraph
  \renewcommand{\subparagraph}[1]{\oldsubparagraph{#1}\mbox{}}
\fi

\usepackage{color}
\usepackage{fancyvrb}
\newcommand{\VerbBar}{|}
\newcommand{\VERB}{\Verb[commandchars=\\\{\}]}
\DefineVerbatimEnvironment{Highlighting}{Verbatim}{commandchars=\\\{\}}
% Add ',fontsize=\small' for more characters per line
\usepackage{framed}
\definecolor{shadecolor}{RGB}{241,243,245}
\newenvironment{Shaded}{\begin{snugshade}}{\end{snugshade}}
\newcommand{\AlertTok}[1]{\textcolor[rgb]{0.68,0.00,0.00}{#1}}
\newcommand{\AnnotationTok}[1]{\textcolor[rgb]{0.37,0.37,0.37}{#1}}
\newcommand{\AttributeTok}[1]{\textcolor[rgb]{0.40,0.45,0.13}{#1}}
\newcommand{\BaseNTok}[1]{\textcolor[rgb]{0.68,0.00,0.00}{#1}}
\newcommand{\BuiltInTok}[1]{\textcolor[rgb]{0.00,0.23,0.31}{#1}}
\newcommand{\CharTok}[1]{\textcolor[rgb]{0.13,0.47,0.30}{#1}}
\newcommand{\CommentTok}[1]{\textcolor[rgb]{0.37,0.37,0.37}{#1}}
\newcommand{\CommentVarTok}[1]{\textcolor[rgb]{0.37,0.37,0.37}{\textit{#1}}}
\newcommand{\ConstantTok}[1]{\textcolor[rgb]{0.56,0.35,0.01}{#1}}
\newcommand{\ControlFlowTok}[1]{\textcolor[rgb]{0.00,0.23,0.31}{#1}}
\newcommand{\DataTypeTok}[1]{\textcolor[rgb]{0.68,0.00,0.00}{#1}}
\newcommand{\DecValTok}[1]{\textcolor[rgb]{0.68,0.00,0.00}{#1}}
\newcommand{\DocumentationTok}[1]{\textcolor[rgb]{0.37,0.37,0.37}{\textit{#1}}}
\newcommand{\ErrorTok}[1]{\textcolor[rgb]{0.68,0.00,0.00}{#1}}
\newcommand{\ExtensionTok}[1]{\textcolor[rgb]{0.00,0.23,0.31}{#1}}
\newcommand{\FloatTok}[1]{\textcolor[rgb]{0.68,0.00,0.00}{#1}}
\newcommand{\FunctionTok}[1]{\textcolor[rgb]{0.28,0.35,0.67}{#1}}
\newcommand{\ImportTok}[1]{\textcolor[rgb]{0.00,0.46,0.62}{#1}}
\newcommand{\InformationTok}[1]{\textcolor[rgb]{0.37,0.37,0.37}{#1}}
\newcommand{\KeywordTok}[1]{\textcolor[rgb]{0.00,0.23,0.31}{#1}}
\newcommand{\NormalTok}[1]{\textcolor[rgb]{0.00,0.23,0.31}{#1}}
\newcommand{\OperatorTok}[1]{\textcolor[rgb]{0.37,0.37,0.37}{#1}}
\newcommand{\OtherTok}[1]{\textcolor[rgb]{0.00,0.23,0.31}{#1}}
\newcommand{\PreprocessorTok}[1]{\textcolor[rgb]{0.68,0.00,0.00}{#1}}
\newcommand{\RegionMarkerTok}[1]{\textcolor[rgb]{0.00,0.23,0.31}{#1}}
\newcommand{\SpecialCharTok}[1]{\textcolor[rgb]{0.37,0.37,0.37}{#1}}
\newcommand{\SpecialStringTok}[1]{\textcolor[rgb]{0.13,0.47,0.30}{#1}}
\newcommand{\StringTok}[1]{\textcolor[rgb]{0.13,0.47,0.30}{#1}}
\newcommand{\VariableTok}[1]{\textcolor[rgb]{0.07,0.07,0.07}{#1}}
\newcommand{\VerbatimStringTok}[1]{\textcolor[rgb]{0.13,0.47,0.30}{#1}}
\newcommand{\WarningTok}[1]{\textcolor[rgb]{0.37,0.37,0.37}{\textit{#1}}}

\providecommand{\tightlist}{%
  \setlength{\itemsep}{0pt}\setlength{\parskip}{0pt}}\usepackage{longtable,booktabs,array}
\usepackage{calc} % for calculating minipage widths
% Correct order of tables after \paragraph or \subparagraph
\usepackage{etoolbox}
\makeatletter
\patchcmd\longtable{\par}{\if@noskipsec\mbox{}\fi\par}{}{}
\makeatother
% Allow footnotes in longtable head/foot
\IfFileExists{footnotehyper.sty}{\usepackage{footnotehyper}}{\usepackage{footnote}}
\makesavenoteenv{longtable}
\usepackage{graphicx}
\makeatletter
\def\maxwidth{\ifdim\Gin@nat@width>\linewidth\linewidth\else\Gin@nat@width\fi}
\def\maxheight{\ifdim\Gin@nat@height>\textheight\textheight\else\Gin@nat@height\fi}
\makeatother
% Scale images if necessary, so that they will not overflow the page
% margins by default, and it is still possible to overwrite the defaults
% using explicit options in \includegraphics[width, height, ...]{}
\setkeys{Gin}{width=\maxwidth,height=\maxheight,keepaspectratio}
% Set default figure placement to htbp
\makeatletter
\def\fps@figure{htbp}
\makeatother

\usepackage{amsmath, xparse}
\usepackage{fancyvrb, fvextra}
\usepackage{unicode-math}
\usepackage{svg}
\usepackage{multicol}
\usepackage{listings}
\usepackage{systeme}
\usepackage{xifthen}
\usepackage{bm}
\DefineVerbatimEnvironment{Highlighting}{Verbatim}{breaklines,commandchars=\\\{\}}
\lstset{basicstyle=\ttfamily\footnotesize,breaklines=true}
\newcommand\rowop[1]{\scriptstyle\smash{\xrightarrow[\vphantom{#1}]{\mkern-4mu#1\mkern-4mu}}}
\DeclareDocumentCommand\converttorows%
{>{\SplitList{,}}m}%
{\ProcessList{#1}{\converttorow}}
\NewDocumentCommand{\converttorow}{m}
{\ifthenelse{\isempty{#1}}{}{\rowop{#1}}\\}

\DeclareDocumentCommand \rowops{m}
{\;
\begin{matrix}
\converttorows {#1}
\end{matrix}
\; }
\KOMAoption{captions}{tableheading}
\makeatletter
\@ifpackageloaded{caption}{}{\usepackage{caption}}
\AtBeginDocument{%
\ifdefined\contentsname
  \renewcommand*\contentsname{Table of contents}
\else
  \newcommand\contentsname{Table of contents}
\fi
\ifdefined\listfigurename
  \renewcommand*\listfigurename{List of Figures}
\else
  \newcommand\listfigurename{List of Figures}
\fi
\ifdefined\listtablename
  \renewcommand*\listtablename{List of Tables}
\else
  \newcommand\listtablename{List of Tables}
\fi
\ifdefined\figurename
  \renewcommand*\figurename{Figure}
\else
  \newcommand\figurename{Figure}
\fi
\ifdefined\tablename
  \renewcommand*\tablename{Table}
\else
  \newcommand\tablename{Table}
\fi
}
\@ifpackageloaded{float}{}{\usepackage{float}}
\floatstyle{ruled}
\@ifundefined{c@chapter}{\newfloat{codelisting}{h}{lop}}{\newfloat{codelisting}{h}{lop}[chapter]}
\floatname{codelisting}{Listing}
\newcommand*\listoflistings{\listof{codelisting}{List of Listings}}
\makeatother
\makeatletter
\makeatother
\makeatletter
\@ifpackageloaded{caption}{}{\usepackage{caption}}
\@ifpackageloaded{subcaption}{}{\usepackage{subcaption}}
\makeatother
\ifLuaTeX
  \usepackage{selnolig}  % disable illegal ligatures
\fi
\usepackage{bookmark}

\IfFileExists{xurl.sty}{\usepackage{xurl}}{} % add URL line breaks if available
\urlstyle{same} % disable monospaced font for URLs
\hypersetup{
  colorlinks=true,
  linkcolor={blue},
  filecolor={Maroon},
  citecolor={Blue},
  urlcolor={Blue},
  pdfcreator={LaTeX via pandoc}}

\author{}
\date{}

\begin{document}

\begin{titlepage}

    \newcommand{\HRule}{\rule{\linewidth}{0.5mm}}
    
    \center
    
    \vspace{10cm}

    \textsc{\LARGE Gwinnett School of Math, Science, and Technology }\\[0.3cm]
    
    \vspace{0.5cm}

    \HRule \\[0.4cm]
    { \huge \bfseries Multivariable Calculus Yearlong Notes}\\[0.03cm]
    \HRule \\[1.5cm]
    
    \begin{minipage}{0.4\textwidth}
    \begin{flushleft} \Large
    Anish Goyal \\1st Period
    \end{flushleft}
    \end{minipage}
    ~
    \begin{minipage}{0.4\textwidth}
    \begin{flushright} \Large
    Donny Thurston\\Educator
    \end{flushright}
    \end{minipage}\\[1cm]
    
    {\huge 2023-2024}\\[1cm]
    
    \includegraphics{img/logo.png}\\
    \vfill
    \end{titlepage}
\newpage

\renewcommand*\contentsname{Table of Contents}
{
\hypersetup{linkcolor=}
\setcounter{tocdepth}{4}
\tableofcontents
}
\newpage{}

\section{Chapter 1: Systems of Linear Equations and
Matrices}\label{chapter-1-systems-of-linear-equations-and-matrices}

\subsection{Matrix Operations}\label{matrix-operations}

\begin{itemize}
\tightlist
\item
  Matrix operations are given as: rows x columns
\item
  Two matrices are equal \(\iff\) they have the same dimensions and
  values
\end{itemize}

\subsubsection{Addition \& Subtraction}\label{addition-subtraction}

Two matrices can be added/subtracted \(\iff\) they have the same
dimensions. \includegraphics{img/addition-subtraction.png}

\subsubsection{Scalar Multiplication}\label{scalar-multiplication}

\begin{itemize}
\tightlist
\item
  Scalar multiplication is defined as multiplying each element of a
  matrix by a number \includegraphics{img/scalar.png}
\end{itemize}

\subsubsection{Matrix Multiplication}\label{matrix-multiplication}

\begin{itemize}
\tightlist
\item
  We can \textbf{only} multiply an (m x n) by (n x p) matrix.
\item
  The resulting matrix will be (m x p)
\end{itemize}

\subsubsection{Properties of Matrix
Arithmetic}\label{properties-of-matrix-arithmetic}

\begin{enumerate}
\def\labelenumi{(\alph{enumi})}
\tightlist
\item
  \(A + B = B + A\) \textbf{(Commutative law for addition)}
\item
  \(A + (B + C) = (A + B) + C\) \textbf{(Associative law for addition)}
\item
  \(A(BC) = (AB)C\) \textbf{(Associative law for multiplication)}
\item
  \(A(B+C) = AB + AC\) \textbf{(Left distributive law)}
\item
  \((B+C)A = BA + CA\) \textbf{(Right distributive law)}
\item
  \(A(B-C) = AB - AC\)
\item
  \((B-C)A = BA - CA\)
\item
  a(B+C) = aB + aC
\item
  a(B-C) = aB - aC
\item
  (a+b)C = aC + bC
\item
  (a-b)C = aC - bC
\item
  a(bC) = (ab)C
\item
  a(BC) = (aB)C = B(aC)
\end{enumerate}

\subsubsection{Examples}\label{examples}

\begin{enumerate}
\def\labelenumi{\arabic{enumi}.}
\item
  \begin{align*}
  &\begin{bmatrix} 1 & 2 \\ 3 & 4\end{bmatrix}\begin{bmatrix} 1 & 2 \\ 3 & 4 \end{bmatrix} \\
  &= \begin{bmatrix} 1 \cdot 1 + 2 \cdot 3 & 1 \cdot 2 + 2 \cdot 4 \\ 3 \cdot 1 + 4 \cdot 3 & 3 \cdot 2 + 4 \cdot 4 \end{bmatrix} \\
  &= \begin{bmatrix} 7 & 10 \\ 15 & 22 \end{bmatrix}
  \end{align*}
\item
  \begin{align*}
  &\begin{bmatrix}2 & -3 \\ 5 & 0 \\ -2 & 4 \\ 1 & 2 \end{bmatrix}\begin{bmatrix}-1 \\ 3 \end{bmatrix} \\
  &=\begin{bmatrix} 2 \cdot (-1) + (-3) \cdot 3 \\ 5 \cdot (-1) + 0 \cdot 3 \\ -2 \cdot (-1) + 4 \cdot 3 \\ 1 \cdot (-1) + 2 \cdot 3 \end{bmatrix} \\
  &=\begin{bmatrix} -11 \\ -5 \\ 14 \\ 5 \end{bmatrix}
  \end{align*}
\item
  \begin{align*}
  &\begin{bmatrix}4 & 5 & -1 \end{bmatrix}\begin{bmatrix} 8 \\ 0 \\ 2\end{bmatrix} \\
  &= \begin{bmatrix} 4 \cdot 8 + 5 \cdot 0 + (-1) \cdot 2 \end{bmatrix} \\
  &= \begin{bmatrix} 30 \end{bmatrix}
  \end{align*}
\end{enumerate}

\subsection{Transpose of a Matrix}\label{transpose-of-a-matrix}

The transpose of an (m x n) matrix is the (n x m) matrix where the rows
and columns are swapped.

If
\(B = \begin{bmatrix} 4 & 2 \\ -1 & 0 \\ 3 & 5 \end{bmatrix}, B^T = \begin{bmatrix} 4 & -1 & 3 \\ 2 & 0 & 5 \end{bmatrix}\)

\begin{align*}
B \cdot B^T &= \begin{bmatrix} 4 & 2 \\ -1 & 0 \\ 3 & 5 \end{bmatrix} \begin{bmatrix} 4 & -1 & 3 \\ 2 & 0 & 5 \end{bmatrix} \\ 
&= \begin{bmatrix} 4 \cdot 4 + 2 \cdot 2 & 4 \cdot (-1) + 2 \cdot 0 & 4 \cdot 3 + 2 \cdot 5 \\ (-1) \cdot 4 + 0 \cdot 2 & (-1) \cdot (-1) + 0 \cdot 0 & (-1) \cdot 3 + 0 \cdot 5 \\ 3 \cdot 4 + 5 \cdot 2 & 3 \cdot (-1) + 5 \cdot 0 & 3 \cdot 3 + 5 \cdot 5 \end{bmatrix} \\ 
&= \begin{bmatrix} 20 & -4 & 22 \\ -4 & 1 & -3 \\ 22 & -3 & 34\end{bmatrix}
\end{align*}

\begin{itemize}
\tightlist
\item
  The transpose of a matrix is \textbf{always} multiplicative with the
  original.
\item
  There is also a \textbf{main diagonal} that is the diagonal from the
  top left to the bottom right, but only square matrices have these.
\item
  The \textbf{trace} of a square matrix \(A\) is equal to the sum of all
  the elements on the main diagonal: \(\mathrm{tr}(A)\)
\end{itemize}

\subsubsection{Transpose Matrix
Properties}\label{transpose-matrix-properties}

\begin{itemize}
\tightlist
\item
  \((A^T)^T = A\)
\item
  \((A + B)^T = A^T + B^T\)
\item
  \((A - B)^T = A^T - B^T\)
\item
  \((kA)^T = kA^T\)
\item
  \((AB)^T = B^T A^T\)
\end{itemize}

\subsection{Homework --- ``Matrix Stuff''
(08/03/2023)}\label{homework-matrix-stuff-08032023}

\subsubsection{\texorpdfstring{Suppose that \(A, B, C, D\) and \(E\) are
matrices with the following
sizes:}{Suppose that A, B, C, D and E are matrices with the following sizes:}}\label{suppose-that-a-b-c-d-and-e-are-matrices-with-the-following-sizes}

\begin{align*}
&A& &B& &C& &D& &E& \\
&(3 \times 2)& &(2 \times 3)& &(3 \times 3)& &(3 \times 2)& &(2 \times 3)&
\end{align*}

For each matrix operation, sort them into undefined if the operation
can't be done, or defined if it can along with the correct dimensions of
the outcome.

\begin{longtable}[]{@{}
  >{\raggedright\arraybackslash}p{(\columnwidth - 6\tabcolsep) * \real{0.1279}}
  >{\raggedright\arraybackslash}p{(\columnwidth - 6\tabcolsep) * \real{0.2907}}
  >{\raggedright\arraybackslash}p{(\columnwidth - 6\tabcolsep) * \real{0.2907}}
  >{\raggedright\arraybackslash}p{(\columnwidth - 6\tabcolsep) * \real{0.2907}}@{}}
\toprule\noalign{}
\begin{minipage}[b]{\linewidth}\raggedright
Undefined
\end{minipage} & \begin{minipage}[b]{\linewidth}\raggedright
Defined; (\(4 \times 2\))
\end{minipage} & \begin{minipage}[b]{\linewidth}\raggedright
Defined; (\(5 \times 5\))
\end{minipage} & \begin{minipage}[b]{\linewidth}\raggedright
Defined; (\(5 \times 2\))
\end{minipage} \\
\midrule\noalign{}
\endhead
\bottomrule\noalign{}
\endlastfoot
\(BA\) & \(AC + D\) & \(E(A+B)\) & \((A^T+E)D\) \\
\(AB + B\) & & & \(E(AC)\) \\
\(E^T A\) & & & \\
\(AE + B\) & & & \\
\end{longtable}

\subsubsection{Consider the matrices}\label{consider-the-matrices}

\[
A = \begin{bmatrix}3 & 0 \\ -1 & 2 \\ 1 & 1 \end{bmatrix}, B=\begin{bmatrix}4 & -1 \\ 0 & 2 \end{bmatrix}, C = \begin{bmatrix} 1 & 4 & 2 \\ 3 & 1 & 5 \end{bmatrix}, D = \begin{bmatrix} 1 & 5 & 2 \\ -1 & 0 & 1 \\ 3 & 2 & 4 \end{bmatrix}, E = \begin{bmatrix} 6 & 1 & 3 \\ -1 & 1 & 2 \\ 4 & 1 & 3 \end{bmatrix}
\]

In each part, compute the given expression (where possible).

\begin{enumerate}
\def\labelenumi{\arabic{enumi}.}
\setcounter{enumi}{1}
\item
  \(\symbf{2A^T + C}\) \begin{align*}
  2A^T + C &= 2\begin{bmatrix}3 & 0 \\ -1 & 2 \\ 1 & 1 \end{bmatrix}^T + \begin{bmatrix} 1 & 4 & 2 \\ 3 & 1 & 5 \end{bmatrix} \\ 
  &= 2\begin{bmatrix}3 & -1 & 1 \\ 0 & 2 & 1 \end{bmatrix} + \begin{bmatrix} 1 & 4 & 2 \\ 3 & 1 & 5 \end{bmatrix} \\
  &= \begin{bmatrix} 6 & -2 & 2 \\ 0 & 4 & 2 \end{bmatrix} + \begin{bmatrix} 1 & 4 & 2 \\ 3 & 1 & 5 \end{bmatrix} \\
  &= \begin{bmatrix} 7 & 2 & 4 \\ 3 & 5 & 7 \end{bmatrix}
  \end{align*}
\item
  \(\symbf{B^T + 5C^T}\) \begin{align*}
  B^T + 5C^T &= \begin{bmatrix}4 & -1 \\ 0 & 2 \end{bmatrix}^T + 5\begin{bmatrix} 1 & 4 & 2 \\ 3 & 1 & 5 \end{bmatrix}^T \\
  &= \begin{bmatrix}4 & 0 \\ -1 & 2 \end{bmatrix} + 5\begin{bmatrix} 1 & 3 \\ 4 & 1 \\ 2 & 5 \end{bmatrix} \\
  &= \begin{bmatrix}4 & 0 \\ -1 & 2 \end{bmatrix} + \begin{bmatrix} 5 & 15 \\ 20 & 5 \\ 10 & 25 \end{bmatrix} \\
  &= \text{Undefined}
  \end{align*}
\item
  \(\symbf{2E^T - 3D^T}\) \begin{align*}
  2E^T - 3D^T &= 2\begin{bmatrix} 6 & 1 & 3 \\ -1 & 1 & 2 \\ 4 & 1 & 3 \end{bmatrix}^T - 3\begin{bmatrix} 1 & 5 & 2 \\ -1 & 0 & 1 \\ 3 & 2 & 4 \end{bmatrix}^T \\
  &= 2\begin{bmatrix} 6 & -1 & 4 \\ 1 & 1 & 1 \\ 3 & 2 & 3 \end{bmatrix} - 3\begin{bmatrix} 1 & -1 & 3 \\ 5 & 0 & 2 \\ 2 & 1 & 4 \end{bmatrix} \\
  &= \begin{bmatrix} 12 & -2 & 8 \\ 2 & 2 & 2 \\ 6 & 4 & 6 \end{bmatrix} - \begin{bmatrix} 3 & -3 & 9 \\ 15 & 0 & 6 \\ 6 & 3 & 12 \end{bmatrix} \\
  &= \begin{bmatrix} 9 & -5 & -1 \\ -13 & 2 & -4 \\ 0 & 1 & -6 \end{bmatrix}
  \end{align*}
\item
  \(\symbf{\mathrm{tr}(DE)}\) \begin{align*}
  &\mathrm{tr}(DE) = \mathrm{tr}\left(\begin{bmatrix}1 & 5 & 2 \\ -1 & 0 & 1 \\ 3 & 2 & 4 \end{bmatrix}\begin{bmatrix}6 & 1 & 3 \\ -1 & 1 & 2 \\ 4 & 1 & 3 \end{bmatrix}\right) \\
  &= \mathrm{tr}\left(\begin{bmatrix}1 \cdot 6 + 5 \cdot (-1) + 2 \cdot 4 & 1 \cdot 1 + 5 \cdot 1 + 2 \cdot 1 & 1 \cdot 3 + 5 \cdot 2 + 2 \cdot 3 \\ (-1) \cdot 6 + 0 \cdot (-1) + 1 \cdot 4 & (-1) \cdot 1 + 0 \cdot 1 + 1 \cdot 1 & (-1) \cdot 3 + 0 \cdot 2 + 1 \cdot 3 \\ 3 \cdot 6 + 2 \cdot (-1) + 4 \cdot 4 & 3 \cdot 1 + 2 \cdot 1 + 4 \cdot 1 & 3 \cdot 3 + 2 \cdot 2 + 4 \cdot 3 \end{bmatrix}\right) \\
  &= \mathrm{tr}\left(\begin{bmatrix} 9 & 8 & 19 \\ -2 & 0 & 0 \\ 32 & 9 & 25 \end{bmatrix}\right) \\
  &= 34
  \end{align*}
\end{enumerate}

\section{Intro to Systems}\label{intro-to-systems}

What are we looking for?

Lines: How many possible solutions?

\begin{itemize}
\tightlist
\item
  Infinite solutions
\item
  One solution
\item
  No solutions
\end{itemize}

Planes: How many possible solutions?

\begin{itemize}
\tightlist
\item
  Infinite solutions
\item
  No solutions
\end{itemize}

What does linear actually mean?

\begin{itemize}
\tightlist
\item
  The word linear \emph{really} means that you've got equations with
  variables and \textbf{all} of the variables are degree one.
\item
  This means that there is no limit to the number of dimensions in a
  linear system.
\end{itemize}

\begin{center}
\includegraphics[width=0.8\textwidth,height=\textheight]{img/linearsystems.png}
\end{center}

\subsection{Review: Solve the following
systems}\label{review-solve-the-following-systems}

\begin{enumerate}
\def\labelenumi{\arabic{enumi}.}
\item
  \systeme{2x+y=10, 3x-y=5}

  \begin{align*}
  5x &= 15 \\
  x &= 3 \\
  2(3)+y &= 10 \\
  6 + y &= 10 \\
  y &= 4
  \end{align*}
\item
  \systeme{2x+y=10, 6x+3y=10}

  \begin{align*}
  y &= 10-2x \\
  6x + 3(10-2x) &= 10 \\
  6x + 30 - 6x &= 10 \\
  30 &= 10 \therefore \text { no solution}
  \end{align*}
\item
  \systeme{5x-2y=4, 15x-6y=12}

  \begin{align*}
  0 &= 0 \\
  12 &= 12 \therefore \text{ infinite solutions}
  \end{align*}
\end{enumerate}

\begin{multicols}{2}
  \hypertarget{review-solve-the-following-systems}{%
  \subsubsection{Consistent}\label{consistent}}

  \begin{itemize}
    \tightlist
    \item A system of equations is \textbf{consistent} if it has at least one solution.
  \end{itemize}

\columnbreak

  \hypertarget{review-solve-the-following-systems}{%
  \subsubsection{Inconsistent}\label{inconsistent}}

  \begin{itemize}
    \tightlist
    \item A system of equations is \textbf{inconsistent} if it has no solutions.
  \end{itemize}

\end{multicols}

\newpage{}

\subsection{The Augmented Matrix}\label{the-augmented-matrix}

\Large\systeme{x-y+2z=5,2x-2y+4z=10,3x-3y+6z=15}
\(\longrightarrow \left[\begin{array}{ccc|c}1 & -1 & 2 & 5 \\ 2 & -2 & 4 & 10 \\ 3 & -3 & 6 & 15 \end{array}\right]\)
\normalsize

\subsection{Elementary Row Operations}\label{elementary-row-operations}

\begin{enumerate}
\def\labelenumi{\arabic{enumi}.}
\tightlist
\item
  Interchange 2 rows
\item
  Multiply a row by a non-zero constant
\item
  Add/substract a multiple of one row to/from another row
\end{enumerate}

Doing these things changes the matrix, but it's the same system!

\subsubsection{Example 1\ldots{} again}\label{example-1-again}

\systeme{2x+y=10, 3x-y=5}

\begin{align*}
\Large
\left[\begin{array}{cc|c}2 & 1 & 10 \\ 3 & -1 & 5 \end{array} \right]&\rowops{\frac{1}{2}R_1,}\left[\begin{array}{cc|c}1 & \frac{1}{2} & 5 \\ 3 & -1 & 5 \end{array} \right] \rowops{,R_2-3R_1}\left[\begin{array}{cc|c}1 & \frac{1}{2} & 5 \\ 0 & -\frac{5}{2} & -10 \end{array}\right] \\
&\rowops{,-\frac{2}{5}R_2} \left[\begin{array}{cc|c}1 & \frac{1}{2} & 5 \\ 0 & 1 & 4 \end{array}\right]\rowops{R1-\frac{1}{2}R_2,}\left[\begin{array}{cc|c}1 & 0 & 3 \\ 0 & 1 & 4 \end{array}\right]
\end{align*}

And so\ldots{} \(x=3\) and \(y=4\)!

\subsection{Connection to Matrices}\label{connection-to-matrices}

If we can make a system's matrix look like

\(\left[\begin{array}{ccc|c}1 & 0 & 0 & c_1 \\ 0 & 1 & 0 & c_2 \\ 0 & 0 & 1 & c_3 \end{array}\right]\),

then the solution to the system will be the ordered triple
\((c_1, c_2, c_3)\).

\subsubsection{Example 2: again}\label{example-2-again}

\systeme{2x+y=10, 6x+3y=10}

\begin{align*}
\left[\begin{array}{cc|c}2 & 1 & 10 \\ 6 & 3 & 10 \end{array}\right] \rowops{\frac{1}{2}R1,} \left[\begin{array}{cc|c}1 & \frac{1}{2} & 5 \\ 6 & 3 & 10 \end{array}\right] \rowops{,R2-6R1,} \left[\begin{array}{cc|c}1 & \frac{1}{2} & 5 \\ 0 & 0 & -20 \end{array}\right]
\end{align*}

This is inconsistent, so there is no solution.

\subsubsection{Example 3: again}\label{example-3-again}

\systeme{5x-2y=4, 15x-6y=12}

\begin{align*}
\left[\begin{array}{cc|c}5 & -2 & 4 \\ 15 & -6 & 12\end{array}\right]\rowops{\frac{1}{5}R1,}\left[\begin{array}{cc|c}1 & -\frac{2}{5} & \frac{4}{5} \\ 15 & -6 & 12 \end{array}\right]\rowops{,R2-15R1}\left[\begin{array}{cc|c}1 & -\frac{2}{5} & \frac{4}{5} \\ 0 & 0 & 0 \end{array}\right]
\end{align*}

Since \(0 = 0\), there are infinitely many solutions.

\subsubsection{Example 4: Solve the following
system}\label{example-4-solve-the-following-system}

\systeme{x_1-2x_2 + x_3 = 0, 2x_2-8x_3=8, -4x_1+5x_2+9x_3=-9}

\begin{align*}
&\left[\begin{array}{ccc|c}
1 & -2 & 1 & 0 \\ 0 & 2 & -8 & 8 \\ -4 & 5 & 9 & -9
\end{array}\right]\rowops{R3+4R1,}\left[\begin{array}{ccc|c}1 & -2 & 1 & 0 \\ 0 & 2 & -8 & 8 \\ 0 & -3 & 13 & -9 \end{array}\right]\rowops{,R3+\frac{3}{2}R2,}\left[\begin{array}{ccc|c}1 & -2 & 1 & 0 \\ 0 & 2 & -8 & 8 \\ 0 & 0 & -1 & 3 \end{array}\right] \\
&\rowops{,\frac{1}{2}R_2,}
\left[\begin{array}{ccc|c}
1 & -2 & 1 & 0 \\ 0 & 1 & -4 & 4 \\ 0 & -3 & 13 & -9 
\end{array}\right] \rowops{R_1+2R_2,,R_3+3R_2}\left[\begin{array}{ccc|c}
1 & 0 & -7 & 8 \\ 0 & 1 & -4 & 4 \\ 0 & 0 & 1 & 3
\end{array}\right]\rowops{R1+7R_3,R_2+4R_3,}\left[\begin{array}{ccc|c}
1 & 0 & 0 & 29 \\ 0 & 1 & 0 & 16 \\ 0 & 0 & 1 & 3
\end{array}\right]
\end{align*}

Therefore the solution to \((x_1, x_2, x_3)\) is \((29, 16, 3)\).

\subsubsection{Elementary Row Operations \& REF Homework Problem
(08/08/2023)}\label{elementary-row-operations-ref-homework-problem-08082023}

\systeme{x + y + 2z = 8, -x - 2y + 3z = 1, 3x - 7y + 4z = 10}

\begin{align*}
&\left[\begin{array}{ccc|c}
1 & 1 & 2 & 8 \\
-1 & -2 & 3 & 1 \\
3 & -7 & 4 & 10
\end{array}\right]\rowops{,R_2+R_1,R_3-3R_1}
\left[\begin{array}{ccc|c}
1 & 1 & 2 & 8 \\
0 & -1 & 5 & 9 \\
0 & -10 & -2 & -14
\end{array}\right]\rowops{,-R_2, -R_3}\left[\begin{array}{ccc|c}
1 & 1 & 2 & 8 \\
0 & 1 & -5 & -9 \\
0 & 10 & 2 & 14
\end{array}\right] \\
&\rowops{R_1-R_2,,R_3-10R_2}\left[\begin{array}{ccc|c}
1 & 0 & 7 & 17 \\
0 & 1 & -5 & -9 \\
0 & 0 & 52 & 104
\end{array}\right]\rowops{,,\frac{1}{52}R_3}\left[\begin{array}{ccc|c}
1 & 0 & 7 & 17 \\
0 & 1 & -5 & -9 \\
0 & 0 & 1 & 2
\end{array}\right]\rowops{R_1-7R_3,R_2+5R_3,}\left[\begin{array}{ccc|c}
1 & 0 & 0 & 3 \\
0 & 1 & 0 & 1 \\
0 & 0 & 1 & 2
\end{array}\right]
\end{align*}

Therefore, the solution to \((x, y, z)\) is \((3, 1, 2)\).

\subsection{Gaussian Elimination}\label{gaussian-elimination}

\ul{Vocabulary}: A matrix is in \ul{Row Echelon Form (REF)} if:

\begin{enumerate}
\def\labelenumi{(\alph{enumi})}
\tightlist
\item
  Any rows of all zeroes are placed at the bottom of the matrix
\item
  All other rows have a \ul{leading 1} (``pivot'')
\item
  As we move down the matrix, each leading 1 is further to the right
  than the 1 above it
\end{enumerate}

A matrix is in \ul{Row Reduced Echelon Form} if the three above
conditions are met in adition to:

\begin{enumerate}
\def\labelenumi{(\alph{enumi})}
\setcounter{enumi}{3}
\tightlist
\item
  Each column with a leading 1 has all other entries in the column as a
  0. (``pivot column'')
\end{enumerate}

\newpage{}

\subsubsection{Examples}\label{examples-1}

\begin{multicols}{2}
$\begin{bmatrix}1 & 0 & 0 & 0 & 8\\ 0 & 1 & 0 & 6 & -3 \\ 0 & 0 & 1 & 7 & 10\\ 0 & 0 & 0 & 0 & 0 \end{bmatrix}$

$\begin{bmatrix}1 & 1 & 1 & 0 \\ 0 & 1 & 1 & 0 \\ 0 & 0 & 0 & 1 \end{bmatrix}$

$\begin{bmatrix}1 & 2 & -3 & 4 \\ 0 & 0 & 0 & 0 \\ 0 & 1 & 2 & -4\end{bmatrix}$
\columnbreak

REF? \checkmark \\
RREF? \checkmark \\

\vspace{1cm}

REF? \checkmark \\
RREF? \times \\

\vspace{1cm}

REF? \times \\
RREF? \times

\end{multicols}

\subsection{\texorpdfstring{Gaussian Elimination With
\textbf{Back-Substitution}}{Gaussian Elimination With Back-Substitution}}\label{gaussian-elimination-with-back-substitution}

\subsubsection{Goal:}\label{goal}

To get the augmented matrix in REF

Solve:
\systeme{x_1 - 2x_2 + 3x_3 = 9, -x_1 + 3x_2 = -4, 2x_1 - 5x_2 + 5x_3 = 17}
\begin{align*}
&\left[\begin{array}{ccc|c}1 & -2 & 3 & 9 \\ -1 & 3 & 0 & -4 \\ 2 & -5 & 5 & 17 \end{array}\right]\rowops{,R_2+R_1,R_3-2R_1}\left[\begin{array}{ccc|c}1 & -2 & 3 & 9 \\ 0 & 1 & 3 & 5 \\ 0 & -1 & -1 & -1 \end{array}\right]\rowops{R_1+2R_2,,R_3+R_2}\left[\begin{array}{ccc|c}1 & 0 & 9 & 19 \\ 0 & 1 & 3 & 5 \\ 0 & 0 & 2 & 4 \end{array}\right] \\
&\rowops{,,\frac{1}{2}R_3}\left[\begin{array}{ccc|c}1 & 0 & 9 & 19 \\ 0 & 1 & 3 & 5 \\ 0 & 0 & 1 & 2\end{array}\right]
\end{align*} \begin{align*}
x + 9z &= 19 \\
y + 3z &= 5 \\
z &= 2 \\
\therefore \ z &= 2, y = 5-3z, x = 19-9z \\
z &= 2, y = 5-3(2), x = 19-9(2) \\
z &= 2, y = -1, x = 1
\end{align*} Therefore, the solution \((x_1, x_2, x_3)\) is
\((1, -1, 2)\).

\subsubsection{Gaussian Elimination Homework Problem
(08/09/2023)}\label{gaussian-elimination-homework-problem-08092023}

\systeme{y+z-2w=-3, x+2y-z=2,2x+4y+z-3w=-2,x-4y-7z-w=-19}

\begin{align*}
&\left[\begin{array}{cccc|c}-2 & 0 & 1 & 1 & -3 \\ 0 & 1 & 2 & -1 & 2 \\ -3 & 2 & 4 & 1 & -2 \\ -1 & 1 & -4 & -7 & -19\end{array}\right]\rowops{R_4,,,R_1}\left[\begin{array}{cccc|c} -1 & 1 & -4 & -7 & -19 \\ 0 & 1 & 2 & -1 & 2 \\ -3 & 2 & 4 & 1 & -2 \\ -2 & 0 & 1 & 1 & -3\end{array}\right]\rowops{-R_1,,,} \\
&\left[\begin{array}{cccc|c} 1 & -1 & 4 & 7 & 19 \\  0 & 1 & 2 & -1 & 2 \\ -3 & 2 & 4 & 1 & -2 \\ -2 & 0 & 1 & 1 & -3\end{array}\right]\rowops{,,R_3+3R_1,R_4+2R_1}\left[\begin{array}{cccc|c} 1 & -1 & 4 & 7 & 19 \\ 0 & 1 & 2 & -1 & 2 \\ 0 & -1 & 16 & 22 & 55 \\ 0 & -2 & 9 & 15 & 35\end{array}\right]\rowops{R_1+R_2,,R_3+R_2,R_4+2R_2} \\
&\left[\begin{array}{cccc|c}1 & 0 & 6 & 6 & 21 \\ 0 & 1 & 2 & -1 & 2 \\ 0 & 0 & 18 & 21 & 57 \\ 0 & 0 & 13 & 13 & 39 \end{array}\right]\rowops{,,\frac{1}{18}R_3,}\left[\begin{array}{cccc|c}1 & 0 & 6 & 6 & 21 \\ 0 & 1 & 2 & -1 & 2 \\ 0 & 0 & 1 & \frac{7}{6} & \frac{19}{6} \\ 0 & 0 & 13 & 13 & 39 \end{array}\right]\rowops{R_1-6R_3,R_2-2R_3,,R_4-13R_3} \\
&\left[\begin{array}{cccc|c}1 & 0 & 0 & -1 & 2 \\ 0 & 1 & 0 & -\frac{10}{3} & -\frac{13}{3} \\ 0 & 0 & 1 & \frac{7}{6} & \frac{19}{6} \\ 0 & 0 & 0 & -\frac{13}{6} & -\frac{13}{6} \end{array}\right]\rowops{,,,\frac{-6}{13}R_4}\left[\begin{array}{cccc|c}1 & 0 & 0 & -1 & 2 \\ 0 & 1 & 0 & -\frac{10}{3} & -\frac{13}{3} \\ 0 & 0 & 1 & \frac{7}{6} & \frac{19}{6} \\ 0 & 0 & 0 & 1 & 1 \end{array}\right]\rowops{R_1+R_4,R_2+\frac{10}{3}R_4,R_3-\frac{7}{6}R_4,} \\
&\left[\begin{array}{cccc|c}1 & 0 & 0 & 0 & 3 \\ 0 & 1 & 0 & 0 & -1 \\ 0 & 0 & 1 & 0 & 2 \\ 0 & 0 & 0 & 1 & 1 \end{array}\right] \Longrightarrow \systeme*{w=3, x=-1, y=2, z=1}
\end{align*}

\newpage{}

\subsection{Gauss-Jordan Elimination}\label{gauss-jordan-elimination}

\subsubsection{Goal:}\label{goal-1}

To get the matrix into RREF

Solve: \systeme{x_1-3x_3=-2, 3x_1+x_2-2x_3=5, 2x_1+2x_2+x_3=4}

\begin{align*}
&\left[\begin{array}{ccc|c}1 & 0 & -3 & -2 \\ 3 & 1 & -2 & 5 \\ 2 & 2 & 1 & 4\end{array}\right]\rowops{,R_2-3R_1,R_3-2R_1}\left[\begin{array}{ccc|c}1 & 0 & -3 & -2 \\ 0 & 1 & 7 & 11 \\ 0 & 2 & 7 & 8\end{array}\right]\rowops{,,R_3-2R_2}\left[\begin{array}{ccc|c}1 & 0 & -3 & -2 \\ 0 & 1 & 7 & 11 \\ 0 & 0 & -7 & -14\end{array}\right]\\
&\rowops{,,\frac{-1}{7}R_3}\left[\begin{array}{ccc|c}1 & 0 & -3 & -2 \\ 0 & 1 & 7 & 11 \\ 0 & 0 & 1 & 2\end{array}\right]\rowops{R_1+3R_3,R_2-7R_3,}\left[\begin{array}{ccc|c}1 & 0 & 0 & 4 \\ 0 & 1 & 0 & -3 \\ 0 & 0 & 1 & 2\end{array}\right] \Longrightarrow \systeme*{x_1=4, x_2=-3, x_3=2}
\end{align*}

\subsection{Matrix Properties, Equations, and
Inverses}\label{matrix-properties-equations-and-inverses}

\subsubsection{With Real Numbers}\label{with-real-numbers}

\begin{itemize}
\tightlist
\item
  If \(ab = bc\), then \(a = c\), if \(b \ne 0\)
\item
  If \(ab = 0\), then \(a = 0\) or \(b = 0\), or both
\end{itemize}

\subsubsection{With Matrices}\label{with-matrices}

\begin{itemize}
\tightlist
\item
  If \(AB = AC\), then \(B = C\), if \(A\) is invertible
\item
  If \(AB = [0]\), then \(A = [0]\) or \(B = [0]\), or both
\end{itemize}

\paragraph{Multiply:}\label{multiply}

\begin{align*}
&\begin{bmatrix} 2 & 3 \\ 3 & 5 \end{bmatrix}
\begin{bmatrix} 5 & -3 \\ -3 & 2 \end{bmatrix} \\
&= \begin{bmatrix} 2(5)+3(-3) & 2(-3)+3(2) \\ 3(5)+5(-3) & 3(-3)+5(2) \end{bmatrix} \\
&= \begin{bmatrix} 1 & 0 \\ 0 & 1 \end{bmatrix}
\end{align*}

\subsubsection{Matrix Inverses}\label{matrix-inverses}

\begin{itemize}
\item
  If a matrix has an inverse, it is said to be invertible or
  non-singular.
\item
  If a matrix does not have an inverse, it is said to be singular.
\item
  Every square matrix has a ``special number'' associated with it called
  the \textbf{determinant}.
\item
  For the \(2 \times 2\) matrix
  \(A = \begin{bmatrix}a & b \\ c & d \end{bmatrix}\), the determinant
  is \(ad-bc\)
\item
  \(A^{-1} = \frac{1}{\det{A}}\begin{bmatrix}d & -b \\ -c & a \end{bmatrix}\)
\item
  When \(\det{A} = 0\), the matrix is singular and has no inverse (since
  you cannot divide by zero)
\end{itemize}

Find the inverse of \(A = \begin{bmatrix}4 & 3 \\ 1 & 2 \end{bmatrix}\)

\begin{align*}
&\begin{bmatrix}4 & 3 \\ 1 & 2 \end{bmatrix}^{-1} &= \frac{1}{\det{A}}\begin{bmatrix}2 & -3 \\ -1 & 4 \end{bmatrix} \\
&= \frac{1}{(4)(2)-(3)(1)}\begin{bmatrix}2 & -3 \\ -1 & 4 \end{bmatrix} \\
&= \frac{1}{5}\begin{bmatrix}2 & -3 \\ -1 & 4 \end{bmatrix} \\
&= \begin{bmatrix}\frac{2}{5} & -\frac{3}{5} \\ -\frac{1}{5} & \frac{4}{5} \end{bmatrix}
\end{align*}

\section{Chapter 2: Determinants}\label{chapter-2-determinants}

\subsection{Prior Knowledge:}\label{prior-knowledge}

\(\begin{bmatrix}10 & -4 \\ -3 & -5 \end{bmatrix}=-50-=-62\)
\begin{align*}
&\begin{bmatrix}2 & 4 & 3 \\ -1 & 2 & 3 \\ 3 & 0 & -2 \end{bmatrix} \\
&= ((2\cdot2\cdot-2) + (4\cdot3\cdot3)+(3\cdot-1\cdot0))-((3\cdot2\cdot3)+(0\cdot3\cdot2)+(-2\cdot-1\cdot4)) \\
&= (-8+36+0)-(18+0+8) \\
&= 28-26 \\
&= 2
\end{align*}

\subsection{Minors \& Cofactors}\label{minors-cofactors}

Given a square matrix A, the \ul{minor} of matrix element \(a_{ij}\),
(\(M_{ij}\)) is the determinant of the matrix formed by removing the
\(i^{\text{th}}\) row and \(j^{\text{th}}\) column from matrix A.

The \ul{cofactor} of matrix element
\(a_{ij}, C_{ij}=(-1)^{i+j}\cdot M_{ij}\)

\subsubsection{Example}\label{example}

Let \(\det\begin{bmatrix}2&4&3 \\ -1&2&3 \\ 3&0&-2\end{bmatrix}\). What
is the cofactor of element (1, 1)?

Cofactor checkerboard:\\
\(\begin{bmatrix}+ & - & + \\ - & + & - \\ + & - & +\end{bmatrix}\)

\(M_{11} = \begin{vmatrix}2 & 3 \\ 0 & -2\end{vmatrix} = -4\)\\
\(C_{11} = 1 \cdot -4 = -4\)

Find the minor and cofactor of: \textbackslash{} a) \(a_{21} = -1\)
\begin{align*}
M_{21} = \begin{vmatrix}4 & 3 \\ 0 & -2 \end{vmatrix} = -8 \\
C_{21} = 8
\end{align*} b) \(a_{33} = -2\) \begin{align*}
M_{33} = \begin{vmatrix}2 & 4 \\ -1 & 2 \end{vmatrix} = 8 \\
C_{33} = 8
\end{align*}

\subsection{Cofactor Expansion}\label{cofactor-expansion}

\begin{enumerate}
\def\labelenumi{\arabic{enumi})}
\tightlist
\item
  Pick a row or column
\item
  Multiply every entry in that row or column by it's corresponding
  cofactor
\item
  Add those together. That's it
\end{enumerate}

\(A = \begin{bmatrix} 6 & 7 & -1 \\ 0 & 4 & 1 \\ 2 & 5 & -3\end{bmatrix}\)
\begin{align*}
\det(A) &= 6\left(\begin{vmatrix}4 & 1 \\ 5 & -3\end{vmatrix}\right) + 7\left(-\begin{vmatrix}0 & 1 \\ 2 & -3\end{vmatrix}\right) + -1\left(\begin{vmatrix}0 & 4 \\ 2 & 5 \end{vmatrix}\right) \\
&=6(-17) + 7(2) + (-1(-8)) \\
&= -102+14+8 \\
&=-80
\end{align*}

\subsubsection{Example}\label{example-1}

\(A = \begin{bmatrix} 6 & 4 & 2 \\ 5 & -6 & 1 \\ 0 & 3 & 0\end{bmatrix}\)
\begin{align*}
& 6\begin{vmatrix}-6 & 1 \\ 3 & 0\end{vmatrix}+4\left(-\begin{vmatrix}5&1 \\ 0 & 0\end{vmatrix}\right)+2\begin{vmatrix}5 & -6 \\ 0 & 3\end{vmatrix} \\
&= 6(-3) + 0 + 2(15) \\
&= -18 + 30 \\
&= 12
\end{align*}

\subsubsection{Does the method generalize to 2x2
matrices?}\label{does-the-method-generalize-to-2x2-matrices}

\begin{align*}
&\begin{vmatrix}3 & 5 \\ 7 & 2\end{vmatrix} \\
&=3|2|-5|7| \\
&=6-35 \\
&=-29
\end{align*}

The determinant of a 1x1 matrix is\ldots{} \textbf{itself!}

\subsubsection{Find the determinant of a
4x4}\label{find-the-determinant-of-a-4x4}

\(A=\begin{bmatrix}-3 & 2 & 0 & 8 \\ 2 & 1 & 0 & -4 \\ 5 & -2 & 1 & 5 \\ 2 & 3 & 0 & 6\end{bmatrix}\)

\begin{align*}
&= 0 + 0 + \begin{vmatrix}-3 & 2 & 8 \\ 2 & 1 & -4 \\ 2 & 3 & 6\end{vmatrix}+0 \\
&= -2\begin{vmatrix}2 & 8 \\ 3 & 6 \end{vmatrix} + \begin{vmatrix}-3 & 8 \\ 2 & 6 \end{vmatrix}-\left(-4\begin{vmatrix}-3 & 2 \\ 2 & 3\end{vmatrix}\right) \\
&= 24 -34 -52 \\
&= -62
\end{align*}

\subsection{Theorem}\label{theorem}

If \(A\) is an \(n \times n\) matrix, then regardless of which row or
column of \(A\) is chosen, the number obtained by multiplying the
elements in that row or column by their corresponding cofactors is
\ul{\textbf{always the same}} and is called the \ul{determinant} of
\(A\).

\subsubsection{Example}\label{example-2}

Find the determinant of
\(A=\begin{bmatrix}1 & 0 & 0 & -1 \\ 3 & 1 & 2 & 2 \\ 1 & 0 & -2 & 1 \\ 2 & 0 & 0 & 1\end{bmatrix}\)
\begin{align*}
&1\cdot\begin{vmatrix}1 & 0 & -1 \\ 1 & -2 & 1 \\ 2 & 0 & 1 \end{vmatrix} \\
&=\left(-2\begin{vmatrix}1 & -1 \\ 2 & 1\end{vmatrix}\right) \\
&= -6
\end{align*}

\subsection{Triangular Matrices}\label{triangular-matrices}

Find the determinant of
\(A = \begin{bmatrix}1 & 1 & 1 & 1 \\ 0 & 2 & 2 & 2 \\ 0 & 0 & 3 & 3 \\ 0 & 0 & 0 & 4\end{bmatrix}\)
\begin{align*}
&\begin{vmatrix}2 & 2 & 2 \\ 0 & 3 & 3 \\ 0 & 0 & 4\end{vmatrix} \\
&= 2\begin{vmatrix}3 & 3 \\ 0 & 4\end{vmatrix} \\
&= 2(3\cdot4) \\
&= 2\cdot12 \\
&=24
\end{align*}

If \(A\) is an \(n \times n\) triangular matrix, then \(\det(A)\) is
equal to the product of the elements along the main diagonal.

\subsection{An Important Definition}\label{an-important-definition}

\ul{Elementary Matrix} a matrix that can be obtanied from the
\(n \times n\) identity matrix by performing a single row operation.
\textbackslash{}

Are the following matrices elementary? 1)
\(\begin{bmatrix}1 & 0 \\ -5 & 1\end{bmatrix}+(R_3+5R_1)\) yes 2)
\(\begin{bmatrix}-5 & 1 \\ 1 & 0\end{bmatrix}+(R_1+5R_2)...\) no

\subsection{A Pair of Theorems}\label{a-pair-of-theorems}

\subsubsection{\texorpdfstring{Theorem: If a square matrix \(A\) has a
row of column of zeros, then
\(\det(A) = 0\)}{Theorem: If a square matrix A has a row of column of zeros, then \textbackslash det(A) = 0}}\label{theorem-if-a-square-matrix-a-has-a-row-of-column-of-zeros-then-deta-0}

\subsubsection{\texorpdfstring{Theorem: If \(A\) is a square matrix,
then
\(\det(A) = \det(A^T)\)}{Theorem: If A is a square matrix, then \textbackslash det(A) = \textbackslash det(A\^{}T)}}\label{theorem-if-a-is-a-square-matrix-then-deta-detat}

\newpage{}

\subsection{Unit 1 \& 2 Homework
Problems}\label{unit-1-2-homework-problems}

\subsubsection{``Gaussian Elimination''
(08/11/2023)}\label{gaussian-elimination-08112023}

\paragraph{Solve this system using Gaussian
Elimination}\label{solve-this-system-using-gaussian-elimination}

\systeme{x_1 + x_2 + 2x_3 = 8, -x_1-2x_2+3x_3 = 1, 3x_1 - 7x_2 +4x_3=10}

\begin{equation*}
\begin{aligned}
&\Rightarrow \left[\begin{array}{ccc|c}1 & 1 & 2 & 8 \\ -1 & -2 & 3 & 1 \\ 3 & -7 & 4 & 10\end{array}\right] \rowops{,R_2+R_1,R_3-3R_1}\left[\begin{array}{ccc|c}1 & 1 & 2 & 8 \\ 0 & -1 & 5 & 9 \\ 0 & -10 & -2 & -14\end{array}\right] \rowops{,-R_2,} \left[\begin{array}{ccc|c}1 & 1 & 2 & 8 \\ 0 & 1 & -5 & -9 \\ 0 & -10 & -2 & -14\end{array}\right] \\
&\rowops{,,R_3+10R_2} \left[\begin{array}{ccc|c}1 & 1 & 2 & 8 \\ 0 & 1 & -5 & -9 \\ 0 & 0 & -52 & -104\end{array}\right]\rowops{,,-\frac{1}{52}R_3}\left[\begin{array}{ccc|c}1 & 1 & 2 & 8 \\ 0 & 1 & -5 & -9 \\ 0 & 0 & 1 & 2\end{array}\right]
\end{aligned}
\end{equation*} \(\therefore\)
\systeme{x_1+x_2+2x_3=8, x_2-5x_3=-9, x_3=2} \(\Rightarrow\)
\systeme*{x_1=3, x_2=1, x_3=2}

\paragraph{Solve this system using Gaussian
Elimination}\label{solve-this-system-using-gaussian-elimination-1}

\systeme{x_1-2x_2+3x_3=0,-2x_1-3x_2-4x_3=0,2x_1-4x_2+4x_3=0}

\begin{equation*}
\begin{aligned}
\Rightarrow \left[\begin{array}{ccc|c}1 & -2 & 3 & 0 \\ -2 & -3 & -4 & 0 \\ 2 & -4 & 4 & 0\end{array}\right] \rowops{,R_2+2R_1,R_3-2R_1}\left[\begin{array}{ccc|c}1 & -2 & 3 & 0 \\ 0 & -7 & 2 & 0 \\ 0 & 0 & -2 & 0\end{array}\right] \rowops{,-\frac{1}{7}R_2,-\frac{1}{2}R_3}\left[\begin{array}{ccc|c}1 & -2 & 3 & 0 \\ 0 & 1 & \frac{2}{7} & 0 \\ 0 & 0 & 1 & 0\end{array}\right]
\end{aligned}
\end{equation*} \(\therefore\)
\systeme*{x_1-2x_2+3x_3=0, x_2+\frac{2}{7}x_3=0, x_3=0}
\(\Rightarrow 1 \ne 0 \therefore \text{ no solution}\)

\newpage{}

\subsubsection{``Inverses and Determinants''
(08/14)}\label{inverses-and-determinants-0814}

\paragraph{Find the determinants of the
following:}\label{find-the-determinants-of-the-following}

\begin{enumerate}
\def\labelenumi{\arabic{enumi})}
\tightlist
\item
  \(\begin{bmatrix}2 & -3 \\ 4 & 4 \end{bmatrix}\)
\end{enumerate}

\(\begin{vmatrix}2 & -3 \\ 4 & 4 \end{vmatrix} = 2(4)-(-3)(4) = 8+12 = 20\)

\begin{enumerate}
\def\labelenumi{\arabic{enumi})}
\setcounter{enumi}{1}
\tightlist
\item
  \(\begin{bmatrix}2 & 0 \\ 0 & 3 \end{bmatrix}\)
\end{enumerate}

\(\begin{vmatrix}2 & 0 \\ 0 & 3 \end{vmatrix} = 2(3)-0(0) = 6\)

\begin{enumerate}
\def\labelenumi{\arabic{enumi})}
\setcounter{enumi}{2}
\tightlist
\item
  \(\begin{bmatrix}\cos\theta & \sin\theta \\ -\sin\theta & \cos\theta \end{bmatrix}\)
\end{enumerate}

\(\begin{vmatrix}\cos\theta & \sin\theta \\ -\sin\theta & \cos\theta \end{vmatrix} = \cos^2\theta + \sin^2\theta = 1\)

\paragraph{Find the INVERSES of those
matrices:}\label{find-the-inverses-of-those-matrices}

\begin{enumerate}
\def\labelenumi{\arabic{enumi})}
\tightlist
\item
  \(\begin{bmatrix}2 & -3 \\ 4 & 4 \end{bmatrix}\)
\end{enumerate}

\(\begin{bmatrix}2 & -3 \\ 4 & 4 \end{bmatrix}^{-1} = \frac{1}{20}\begin{bmatrix}4 & 3 \\ -4 & 2 \end{bmatrix}=\begin{bmatrix}\frac{1}{5} & \frac{3}{20} \\ -\frac{1}{5} & \frac{1}{10}\end{bmatrix}\)

\begin{enumerate}
\def\labelenumi{\arabic{enumi})}
\setcounter{enumi}{1}
\tightlist
\item
  \(\begin{bmatrix}2 & 0 \\ 0 & 3 \end{bmatrix}\)
\end{enumerate}

\(\begin{bmatrix}2 & 0 \\ 0 & 3 \end{bmatrix}^{-1} = \frac{1}{6}\begin{bmatrix}3 & 0 \\ 0 & 2 \end{bmatrix}=\begin{bmatrix}\frac{1}{2} & 0 \\ 0 & \frac{1}{3}\end{bmatrix}\)

\begin{enumerate}
\def\labelenumi{\arabic{enumi})}
\setcounter{enumi}{2}
\tightlist
\item
  \(\begin{bmatrix}\cos\theta & \sin\theta \\ -\sin\theta & \cos\theta \end{bmatrix}\)
\end{enumerate}

\(\begin{bmatrix}\cos\theta & \sin\theta \\ -\sin\theta & \cos\theta \end{bmatrix}^{-1} = \begin{bmatrix}\cos\theta & -\sin\theta \\ \sin\theta & \cos\theta \end{bmatrix}\)

\subsubsection{Inverses and Determinants
(08/15)}\label{inverses-and-determinants-0815}

\paragraph{Use a matrix equation to solve the following
problems:}\label{use-a-matrix-equation-to-solve-the-following-problems}

\begin{enumerate}
\def\labelenumi{\arabic{enumi})}
\item
  \systeme{3x_1-2x_2=1,4x_1+5x_2=3}
\end{enumerate}

\begin{align*}
&\Rightarrow \begin{bmatrix}3 & -2 \\ 4 & 5\end{bmatrix}\begin{bmatrix}x_1 \\ x_2\end{bmatrix}=\begin{bmatrix}-1 \\ 3\end{bmatrix} \\
& \begin{bmatrix}x_1 \\ x_2\end{bmatrix}=\begin{bmatrix}3 & -2 \\ 4 & 5\end{bmatrix}^{-1}\begin{bmatrix}-1 \\ 3\end{bmatrix} \\
& \begin{bmatrix}x_1 \\ x_2\end{bmatrix}=\frac{1}{23}\begin{bmatrix}5 & 2 \\ -4 & 3\end{bmatrix}\begin{bmatrix}-1 \\ 3\end{bmatrix} \\
& \begin{bmatrix}x_1 \\ x_2\end{bmatrix}=\frac{1}{23}\begin{bmatrix}-1 \\ 9\end{bmatrix} \\
& \begin{bmatrix}x_1 \\ x_2\end{bmatrix}=\begin{bmatrix}-\frac{1}{23} \\ \frac{9}{23}\end{bmatrix}
\end{align*}

\begin{enumerate}
\def\labelenumi{\arabic{enumi})}
\setcounter{enumi}{1}
\item
  \systeme{6x_1+x_2=0,4x_1-3x_2=-2}
\end{enumerate}

\begin{align*}
&\Rightarrow \begin{bmatrix}6 & 1 \\ 4 & -3\end{bmatrix}\begin{bmatrix}x_1 \\ x_2\end{bmatrix}=\begin{bmatrix}0 \\ -2\end{bmatrix} \\
&\begin{bmatrix}x_1 \\ x_2\end{bmatrix}=\begin{bmatrix}6 & 1 \\ 4 & -3\end{bmatrix}^{-1}\begin{bmatrix}0 \\ -2\end{bmatrix} \\
&\begin{bmatrix}x_1 \\ x_2\end{bmatrix}=\frac{1}{-22}\begin{bmatrix}-3 & -1 \\ -4 & 6\end{bmatrix}\begin{bmatrix}0 \\ -2\end{bmatrix} \\
&\begin{bmatrix}x_1 \\ x_2\end{bmatrix}=\frac{1}{-22}\begin{bmatrix}2 \\ 8\end{bmatrix} \\
&\begin{bmatrix}x_1 \\ x_2\end{bmatrix}=\begin{bmatrix}-\frac{1}{11} \\ -\frac{4}{11}\end{bmatrix}
\end{align*}

\newpage{}

\subsubsection{Consistent Systems
(08/21)}\label{consistent-systems-0821}

\paragraph{Solve the linear systems together by reducing the appropriate
augmented
matrix.}\label{solve-the-linear-systems-together-by-reducing-the-appropriate-augmented-matrix.}

\systeme{x_1-5x_2=b_1,3x_1+2x_2=b_2}

\begin{enumerate}
\def\labelenumi{\arabic{enumi})}
\tightlist
\item
  \(b_1=1, \ b_2=4\)
\item
  \(b_1=-2, \ b_2=5\)
\end{enumerate}

First, let's solve it for the general case: \begin{align*}
&\left[\begin{array}{cc|c}1 & -5 & b_1 \\ 3 & 2 & b_2\end{array}\right] \rowops{,R_2-3R_1}\left[\begin{array}{cc|c}1 & -5 & b_1 \\ 0 & 17 & b_2-3b_1\end{array}\right] \rowops{,\frac{1}{17}R_2}\left[\begin{array}{cc|c}1 & -5 & b_1 \\ 0 & 1 & \frac{b_2-3b_1}{17}\end{array}\right] \rowops{R_1+5R_2,}\left[\begin{array}{cc|c}1 & 0 & \frac{2b_1+5b_2}{17} \\ 0 & 1 & \frac{-3b_1+b_2}{17}\end{array}\right]
\end{align*} Therefore, the solution to the general case is
\((x_1, x_2) = (\frac{2b_1+5b_2}{17}, \frac{-3b_1+b_2}{17})\)

And so, for the specific cases:

\begin{enumerate}
\def\labelenumi{\arabic{enumi})}
\tightlist
\item
  \((x_1, x_2) = \left(\frac{2(1)+5(4)}{17}, \frac{-3(1)+4}{17}\right) = \left(\frac{13}{17}, \frac{1}{17}\right)\)
\item
  \((x_1, x_2) = \left(\frac{2(-2)+5(5)}{17}, \frac{-3(-2)+5}{17}\right) = \left(\frac{16}{17}, \frac{11}{17}\right)\)
\end{enumerate}

\newpage{}

\paragraph{\texorpdfstring{Determine the conditions on \(b\), if any, in
order to guarantee that the linear system is
consistent.}{Determine the conditions on b, if any, in order to guarantee that the linear system is consistent.}}\label{determine-the-conditions-on-b-if-any-in-order-to-guarantee-that-the-linear-system-is-consistent.}

\systeme{x_1+3x_2=b_1,-2x_1+x_2=b_2}

\begin{align*}
&\left[\begin{array}{cc|c}1 & 3 & b_1 \\ -2 & 1 & b_2\end{array}\right] \rowops{,R_2+2R_1}\left[\begin{array}{cc|c}1 & 3 & b_1 \\ 0 & 7 & b_2+2b_1\end{array}\right] \rowops{,\frac{1}{7}R_2}\left[\begin{array}{cc|c}1 & 3 & b_1 \\ 0 & 1 & \frac{b_2+2b_1}{7}\end{array}\right] \rowops{R_1-3R_2,}\left[\begin{array}{cc|c}1 & 0 & \frac{b_1-3b_2}{7} \\ 0 & 1 & \frac{b_2+2b_1}{7}\end{array}\right]
\end{align*} There are no conditions. The system is consistent for all
values of \(b_1\) and \(b_2\).

\subsubsection{Another ``determining the conditions''
problem:}\label{another-determining-the-conditions-problem}

\systeme{x_1-2x_2-x_3=b_1, -4x_1+5x_2+2x_3=b_2, -4x_1+7x_2+4x_3=b_3}

\begin{align*}
&\left[\begin{array}{ccc|c}1 & -2 & -1 & b_1 \\ -4 & 5 & 2 & b_2 \\ -4 & 7 & 4 & b_3 \end{array}\right] \rowops{,R_2+4R_1,R_3+4R_1}\left[\begin{array}{ccc|c}1 & -2 & -1 & b_1 \\ 0 & -3 & -2 & b_2+4b_1 \\ 0 & -1 & 0 & b_3+4b_1\end{array}\right] \rowops{,-\frac{1}{3}R_2,}\left[\begin{array}{ccc|c}1 & -2 & -1 & b_1 \\ 0 & 1 & \frac{2}{3} & \frac{-b_2-4b_1}{3} \\ 0 & 0 & -\frac{2}{3} & \frac{b_3+4b_1}{3}\end{array}\right]\\
&\rowops{,,-\frac{3}{2}R_3}\left[\begin{array}{ccc|c}1 & -2 & -1 & b_1 \\ 0 & 1 & \frac{2}{3} & \frac{-b_2-4b_1}{3} \\ 0 & 0 & 1 & \frac{-b_3-4b_1}{2}\end{array}\right]
\end{align*}

Therefore, the system is consistent for all values of \(b_1\), \(b_2\),
and \(b_3\).

\newpage{}

\subsubsection{Triangular and Diagonal
Matrices}\label{triangular-and-diagonal-matrices}

\paragraph{\texorpdfstring{Find \(A^2\)}{Find A\^{}2}}\label{find-a2}

\begin{enumerate}
\def\labelenumi{\arabic{enumi})}
\tightlist
\item
  \(A=\begin{bmatrix}1 & 0 \\ 0 & -2\end{bmatrix}\) \begin{align*}
  A^2 &= \begin{bmatrix}1 & 0 \\ 0 & -2\end{bmatrix}\begin{bmatrix}1 & 0 \\ 0 & -2\end{bmatrix} \\
  &= \begin{bmatrix}1(1)+0(0) & 1(0)+0(-2) \\ 0(1)+(-2)(0) & 0(0)+(-2)(-2)\end{bmatrix} \\
  &= \begin{bmatrix}1 & 0 \\ 0 & 4\end{bmatrix}
  \end{align*}
\item
  \(A=\begin{bmatrix}-6 & 0 & 0 \\ 0 & 3 & 0 \\ 0 & 0 & 5 \end{bmatrix}\)
  \begin{align*}
  A^2 &= \begin{bmatrix}-6 & 0 & 0 \\ 0 & 3 & 0 \\ 0 & 0 & 5 \end{bmatrix}\begin{bmatrix}-6 & 0 & 0 \\ 0 & 3 & 0 \\ 0 & 0 & 5 \end{bmatrix} \\
  &= \begin{bmatrix}(-6)(-6)+(0)(0)+(0)(0) & (-6)(0)+(0)(3)+(0)(0) & (-6)(0)+(0)(0)+(0)(5) \\ (0)(-6)+(3)(0)+(0)(0) & (0)(0)+(3)(3)+(0)(0) & (0)(0)+(3)(0)+(0)(5) \\ (0)(-6)+(0)(0)+(5)(0) & (0)(0)+(0)(3)+(5)(0) & (0)(0)+(0)(0)+(5)(5)\end{bmatrix} \\
  &= \begin{bmatrix}36 & 0 & 0 \\ 0 & 9 & 0 \\ 0 & 0 & 25\end{bmatrix}
  \end{align*}
\end{enumerate}

\newpage{}

\paragraph{\texorpdfstring{Find \(A^{-k}\), such that \(k\) is some
nonzero
constant}{Find A\^{}\{-k\}, such that k is some nonzero constant}}\label{find-a-k-such-that-k-is-some-nonzero-constant}

\(A=\begin{bmatrix}2 & 0 & 0 & 0 \\ 0 & -4 & 0 & 0 \\ 0 & 0 & -3 & 0 \\ 0 & 0 & 0 & 2 \end{bmatrix}\)
\begin{align*}
A^{-k} &= \begin{bmatrix}2 & 0 & 0 & 0 \\ 0 & -4 & 0 & 0 \\ 0 & 0 & -3 & 0 \\ 0 & 0 & 0 & 2 \end{bmatrix}^{-k} \\
&= \begin{bmatrix}2^{-k} & 0 & 0 & 0 \\ 0 & (-4)^{-k} & 0 & 0 \\ 0 & 0 & (-3)^{-k} & 0 \\ 0 & 0 & 0 & 2^{-k} \end{bmatrix}
\end{align*}

\includegraphics{img/symmetric-or-not.png}

\newpage{}

\paragraph{\texorpdfstring{Find a diagonal matrix \(A\) that satisfies
the given
condition}{Find a diagonal matrix A that satisfies the given condition}}\label{find-a-diagonal-matrix-a-that-satisfies-the-given-condition}

\begin{enumerate}
\def\labelenumi{\arabic{enumi})}
\tightlist
\item
  \(A^5=\begin{bmatrix}1 & 0 & 0 \\ 0 & -1 & 0 \\ 0 & 0 & -1\end{bmatrix}\)
  \begin{align*}
  A &= \begin{bmatrix}1 & 0 & 0 \\ 0 & -1 & 0 \\ 0 & 0 & -1\end{bmatrix}^{\frac{1}{5}} \\
  &= \begin{bmatrix}1^{\frac{1}{5}} & 0 & 0 \\ 0 & (-1)^{\frac{1}{5}} & 0 \\ 0 & 0 & (-1)^{\frac{1}{5}}\end{bmatrix} \\
  &= \begin{bmatrix}1 & 0 & 0 \\ 0 & -1 & 0 \\ 0 & 0 & -1\end{bmatrix}
  \end{align*}
\item
  \(A^{-2}=\begin{bmatrix}9 & 0 & 0 \\ 0 & 4 & 0 \\ 0 & 0 & 1\end{bmatrix}\)
  \begin{align*}
  A &= \begin{bmatrix}9 & 0 & 0 \\ 0 & 4 & 0 \\ 0 & 0 & 1\end{bmatrix}^{-\frac{1}{2}} \\
  &= \begin{bmatrix}9^{-\frac{1}{2}} & 0 & 0 \\ 0 & 4^{-\frac{1}{2}} & 0 \\ 0 & 0 & 1^{-\frac{1}{2}}\end{bmatrix} \\
  &= \begin{bmatrix}\frac{1}{3} & 0 & 0 \\ 0 & \frac{1}{2} & 0 \\ 0 & 0 & 1\end{bmatrix}
  \end{align*}
\end{enumerate}

\newpage{}

\subsubsection{Determinants and Triangular Matrices
(08/29)}\label{determinants-and-triangular-matrices-0829}

\paragraph{\texorpdfstring{What is
\(C_{32}\)}{What is C\_\{32\}}}\label{what-is-c_32}

\(A = \begin{bmatrix}2 & 3 & -1 & 1 \\ -3 & 2 & 0 & 3 \\ 3 & -2 & 1 & 0 \\ 3 & -2 & 1 & 4 \end{bmatrix}\)
\begin{align*}
C_{32} &= (-1)^{3+2}\begin{vmatrix}2 & -1 & 1 \\ -3 & 0 & 3 \\ 3 & 1 & 0\end{vmatrix} \\
&= -\begin{vmatrix}2 & -1 & 1 \\ -3 & 0 & 3 \\ 3 & 1 & 0\end{vmatrix} \\
&= -\left(2\begin{vmatrix}0 & 3 \\ 1 & 0\end{vmatrix}-(-1)\begin{vmatrix}-3 & 3 \\ 3 & 0\end{vmatrix}+1\begin{vmatrix}-3 & 0 \\ 3 & 1\end{vmatrix}\right) \\
&= -\left(2(-3)-(-1)(-9)+1(-3)\right) \\
&= -(-6+9-3) \\
&= 0
\end{align*}

\paragraph{\texorpdfstring{Find all values of \(\lambda\) such that
\textbar{}\(A\)\textbar{} =
0}{Find all values of \textbackslash lambda such that \textbar A\textbar{} = 0}}\label{find-all-values-of-lambda-such-that-a-0}

\(A = \begin{bmatrix}\lambda-2 & 1 \\ -5 & \lambda+4\end{bmatrix}\)
\begin{align*}
\det(A) &= (\lambda-2)(\lambda+4)-(-5)(1) \\
&= \lambda^2+2\lambda-8+5 \\
&= \lambda^2+2\lambda-3 \\
&= (\lambda+3)(\lambda-1) \\
&= 0
\end{align*}

Therefore, \(\lambda = -3, 1\)

\newpage{}

\paragraph{\texorpdfstring{For the matrix
\(\begin{bmatrix}3 & 0 & 0 \\2 & -1 & 5 \\ 1 & 9 & -4\end{bmatrix}\)
find the determinant 3 sp.different ways with cofactor expansion. Pick
sp.different rows and columns each
time.}{For the matrix \textbackslash begin\{bmatrix\}3 \& 0 \& 0 \textbackslash\textbackslash2 \& -1 \& 5 \textbackslash\textbackslash{} 1 \& 9 \& -4\textbackslash end\{bmatrix\} find the determinant 3 sp.different ways with cofactor expansion. Pick sp.different rows and columns each time.}}\label{for-the-matrix-beginbmatrix3-0-0-2--1-5-1-9--4endbmatrix-find-the-determinant-3-sp.different-ways-with-cofactor-expansion.-pick-sp.different-rows-and-columns-each-time.}

\begin{align*}
\det(A) &= 3\begin{vmatrix}-1 & 5 \\ 9 & -4\end{vmatrix}-0\begin{vmatrix}2 & 5 \\ 1 & -4\end{vmatrix}+0\begin{vmatrix}2 & -1 \\ 1 & 9\end{vmatrix} \\
&= 3(-1(-4)-5(9))-0(2(-4)-5(1))+0(2(9)-(-1)(1)) \\
&= 3(4-45)-0(-8-5)+0(18+1) \\
&= 3(-41)-0(-13)+0(19) \\
&= 36
\end{align*}

\begin{align*}
\det(A) &= 0\begin{vmatrix}2 & 5 \\ 9 & -4\end{vmatrix}-3\begin{vmatrix}3 & 0 \\ 1 & -4\end{vmatrix}+0\begin{vmatrix}3 & 0 \\ 2 & 5\end{vmatrix} \\
&= 0(2(-4)-5(9))-3(3(-4)-0(1))+0(3(5)-0(2)) \\
&= 0(-8-45)-3(-12-0)+0(15-0) \\
&= 0(-53)-3(-12) \\
&= 36
\end{align*}

\begin{align*}
\det(A) &= 0\begin{vmatrix}2 & -1 \\ 9 & -4\end{vmatrix}-0\begin{vmatrix}3 & 0 \\ 1 & -4\end{vmatrix}+3\begin{vmatrix}3 & 0 \\ 2 & -1\end{vmatrix} \\
&= 0(2(-4)-(-1)(9))-0(3(-4)-0(1))+3(3(-1)-0(2)) \\
&= 0(-8+9)-0(-12-0)+3(-3-0) \\
&= 0(1)-0(-12)+3(-3) \\
&= 0+0-9 \\
&= 36
\end{align*}

\newpage{}

\paragraph{\texorpdfstring{Evaluate \(\det(A)\) by a cofactor expansion
along a row or column of your
choice}{Evaluate \textbackslash det(A) by a cofactor expansion along a row or column of your choice}}\label{evaluate-deta-by-a-cofactor-expansion-along-a-row-or-column-of-your-choice}

\(A = \begin{bmatrix}1 & k & k^2 \\ 1 & k & k^2 \\ 1 & k & k^2\end{bmatrix}\)
\begin{align*}
\det(A) &= 1\begin{vmatrix}k & k^2 \\ k & k^2\end{vmatrix}-k\begin{vmatrix}1 & k^2 \\ 1 & k^2\end{vmatrix}+k^2\begin{vmatrix}1 & k \\ 1 & k\end{vmatrix} \\
&= 1(k^2-k^2)-k(1(k^2)-k^2(1))+k^2(1(k)-k(1)) \\
&= 0
\end{align*}

\paragraph{Evaluate the determinant of the following matrices by just
looking at
them.}\label{evaluate-the-determinant-of-the-following-matrices-by-just-looking-at-them.}

\(A=\begin{bmatrix}1 & 0 & 0 \\ 0 & -1 & 0 \\ 0 & 0 & 1 \end{bmatrix}\)
\begin{align*}
\det(A) &= 1(-1)(1) = -1
\end{align*}
\(A=\begin{bmatrix}1 & 2 & 7 & -3 \\ 0 & 1 & -4 & 1 \\ 0 & 0 & 2 & 7 \\ 0 & 0 & 0 & 3 \end{bmatrix}\)
\begin{align*}
\det(A) &= 1(1)(2)(3) = 6
\end{align*}

\paragraph{\texorpdfstring{Show that the value of the determinant is
independent of
\(\theta\)}{Show that the value of the determinant is independent of \textbackslash theta}}\label{show-that-the-value-of-the-determinant-is-independent-of-theta}

\(A=\begin{vmatrix}\sin\theta & \cos\theta & 0 \\
-\cos\theta & \sin\theta & 0 \\
\sin\theta-\cos\theta & \sin\theta+\cos\theta & 1\end{vmatrix}\)
\begin{align*}
\det(A) &= \sin\theta\begin{vmatrix}\sin\theta & 0 \\ \sin\theta+\cos\theta & 1\end{vmatrix}-\cos\theta\begin{vmatrix}\cos\theta & 0 \\ \sin\theta+\cos\theta & 1\end{vmatrix}\\ 
+0\begin{vmatrix}\cos\theta & \sin\theta \\ \sin\theta+\cos\theta & \sin\theta\end{vmatrix} \\
&= \sin\theta\left(\sin\theta(1)-0(\sin\theta+\cos\theta)\right)-\cos\theta\left(\cos\theta(1)-0(\sin\theta+\cos\theta)\right) \\
+0\left(\cos\theta(\sin\theta)-\sin\theta(\sin\theta+\cos\theta)\right) \\
&= \sin^2\theta-\cos^2\theta \\
&= 1
\end{align*}

\newpage{}

\subsubsection{Row operations and Determinants
(08/31)}\label{row-operations-and-determinants-0831}

\paragraph{\texorpdfstring{Find the determinant of
\(\begin{bmatrix}1 & -3 & 0 \\ -2 & 4 & 1 \\ 5 & -2 & 2 \end{bmatrix}\)
WITHOUT using cofactor
expansion}{Find the determinant of \textbackslash begin\{bmatrix\}1 \& -3 \& 0 \textbackslash\textbackslash{} -2 \& 4 \& 1 \textbackslash\textbackslash{} 5 \& -2 \& 2 \textbackslash end\{bmatrix\} WITHOUT using cofactor expansion}}\label{find-the-determinant-of-beginbmatrix1--3-0--2-4-1-5--2-2-endbmatrix-without-using-cofactor-expansion}

\begin{align*}
\det(A) &= \begin{vmatrix}1 & -3 & 0 \\ -2 & 4 & 1 \\ 5 & -2 & 2 \end{vmatrix} \\
&= \begin{vmatrix}1 & -3 & 0 \\ 0 & -2 & 1 \\ 0 & 13 & 2 \end{vmatrix} \\
&= \begin{vmatrix}1 & -3 & 0 \\ 0 & -2 & 1 \\ 0 & 0 & \frac{28}{2} \end{vmatrix} \\
&= 1(-2)\left(\frac{28}{2}\right) \\
&= -28
\end{align*}

\newpage{}

\paragraph{\texorpdfstring{Find the determinant of
\(\begin{bmatrix} 2 & 1 & 3 & 1 \\ 1 & 0 & 1 & 1 \\ 0 & 2 & 1 & 0 \\ 0 & 1 & 2 & 3\end{bmatrix}\)}{Find the determinant of \textbackslash begin\{bmatrix\} 2 \& 1 \& 3 \& 1 \textbackslash\textbackslash{} 1 \& 0 \& 1 \& 1 \textbackslash\textbackslash{} 0 \& 2 \& 1 \& 0 \textbackslash\textbackslash{} 0 \& 1 \& 2 \& 3\textbackslash end\{bmatrix\}}}\label{find-the-determinant-of-beginbmatrix-2-1-3-1-1-0-1-1-0-2-1-0-0-1-2-3endbmatrix}

\begin{align*}
\det(A) &= \begin{vmatrix} 2 & 1 & 3 & 1 \\ 1 & 0 & 1 & 1 \\ 0 & 2 & 1 & 0 \\ 0 & 1 & 2 & 3\end{vmatrix} \\
&= \begin{vmatrix} 2 & 1 & 3 & 1 \\ 0 & -2 & -5 & -1 \\ 0 & 2 & 1 & 0 \\ 0 & 1 & 2 & 3\end{vmatrix} \\
&= \begin{vmatrix} 2 & 1 & 3 & 1 \\ 0 & -2 & -5 & -1 \\ 0 & 0 & -4 & -1 \\ 0 & 0 & -3 & 2\end{vmatrix} \\
&= 2(-2)(-4)(2) \\
&= 64
\end{align*}

\newpage{}

\subsubsection{Adjoints and Cramer's Rule
(09/05)}\label{adjoints-and-cramers-rule-0905}

\paragraph{\texorpdfstring{Find the inverse of
\(A=\begin{bmatrix}2 & 5 & 5 \\ -1 & -1 & 0 \\ 2 & 4 & 3\end{bmatrix}\)
using the adjoint
method}{Find the inverse of A=\textbackslash begin\{bmatrix\}2 \& 5 \& 5 \textbackslash\textbackslash{} -1 \& -1 \& 0 \textbackslash\textbackslash{} 2 \& 4 \& 3\textbackslash end\{bmatrix\} using the adjoint method}}\label{find-the-inverse-of-abeginbmatrix2-5-5--1--1-0-2-4-3endbmatrix-using-the-adjoint-method}

\begin{align*}
\det(A) &= 2\begin{vmatrix}-1 & 0 \\ 4 & 3\end{vmatrix}-5\begin{vmatrix}-1 & 0 \\ 2 & 3\end{vmatrix}+5\begin{vmatrix}-1 & -1 \\ 2 & 4\end{vmatrix} \\
&= 2(-3)-5(-3)+5(-2) \\
&= -6+15-10 \\
&= -1 \\
\text{adj}(A) &= \begin{bmatrix}(-1)^{1+1}\begin{vmatrix}-1 & 0 \\ 4 & 3\end{vmatrix} & (-1)^{1+2}\begin{vmatrix}-1 & 0 \\ 2 & 3\end{vmatrix} & (-1)^{1+3}\begin{vmatrix}-1 & -1 \\ 2 & 4\end{vmatrix} \\ (-1)^{2+1}\begin{vmatrix}5 & 5 \\ 4 & 3\end{vmatrix} & (-1)^{2+2}\begin{vmatrix}2 & 5 \\ 2 & 3\end{vmatrix} & (-1)^{2+3}\begin{vmatrix}2 & 5 \\ 2 & 4\end{vmatrix} \\ (-1)^{3+1}\begin{vmatrix}5 & 5 \\ -1 & 0\end{vmatrix} & (-1)^{3+2}\begin{vmatrix}2 & 5 \\ -1 & 0\end{vmatrix} & (-1)^{3+3}\begin{vmatrix}2 & 5 \\ -1 & -1\end{vmatrix}\end{bmatrix} \\
&= \begin{bmatrix}(-1)(3) & -(-1)(3) & -4+2 \\ -(15-20) & 6-10 & -(8-10) \\ 5 & -5 & -2+5\end{bmatrix}^T \\
&= \begin{bmatrix}-3 & 3 & -2 \\ 5 & -4 & 2 \\ 5 & -5 & 3\end{bmatrix}^T \\
&= \begin{bmatrix}-3 & 5 & 5 \\ 3 & -4 & -5 \\ -2 & 2 & 3\end{bmatrix} \\
\therefore \ A^{-1} &= -\begin{bmatrix}-3 & 5 & 5 \\ 3 & -4 & -5 \\ -2 & 2 & 3\end{bmatrix} \\
&= \begin{bmatrix}3 & -5 & -5 \\ -3 & 4 & 5 \\ 2 & -2 & -3\end{bmatrix}
\end{align*}

\newpage{}

\paragraph{Solve the following system of equations using Cramer's
Rule}\label{solve-the-following-system-of-equations-using-cramers-rule}

\systeme{4x+5y=2,11x+y+2z=3,x+5y+2z=1}
\(\longrightarrow \begin{vmatrix} 4 & 5 & 0 \\ 11 & 1 & 2 \\ 1 & 5 & 2\end{vmatrix} \longrightarrow 4\begin{vmatrix}1 & 2 \\ 5 & 2\end{vmatrix}-5\begin{vmatrix}11 & 2 \\ 1 & 2\end{vmatrix} = -132\)
\begin{align*}
\det{(x)} &= \begin{vmatrix}2 & 5 & 0 \\ 3 & 1 & 2 \\ 1 & 5 & 2\end{vmatrix} \\
&= 2\begin{vmatrix}1 & 2 \\ 5 & 2\end{vmatrix}-5\begin{vmatrix}3 & 2 \\ 1 & 2\end{vmatrix} \\
&= 2(2-10)-5(6-2) \\
&= -16-20 \\
&= -36 \\
\det{(y)} &= \begin{vmatrix}4 & 2 & 0 \\ 11 & 3 & 2 \\ 1 & 1 & 2\end{vmatrix} \\
&= 4\begin{vmatrix}3 & 2 \\ 1 & 2\end{vmatrix}-2\begin{vmatrix}11 & 2 \\ 1 & 2\end{vmatrix} \\
&= 4(6-2)-2(22-2) \\
&= 16-40 \\
&= -24 \\
\det{(z)} &= \begin{vmatrix}4 & 5 & 2 \\ 11 & 1 & 3 \\ 1 & 5 & 1\end{vmatrix} \\
&= 4\begin{vmatrix}1 & 3 \\ 5 & 1\end{vmatrix}-5\begin{vmatrix}11 & 3 \\ 1 & 3\end{vmatrix}+2\begin{vmatrix}11 & 1 \\ 1 & 5\end{vmatrix} \\
&= 4(1-15)-5(33-3)+2(55-1) \\
&= -56-150+108 \\
&= -98 \\
\end{align*}

Therefore, the solution
\((x, y, z) = (\frac{3}{11}, \frac{2}{11}, -\frac{49}{66})\)

\newpage{}

\section{Chapter 5: Eigenvectors and
Eigenvalues}\label{chapter-5-eigenvectors-and-eigenvalues}

\subsection{Eigenvalues and Eigenvectors
(11/06)}\label{eigenvalues-and-eigenvectors-1106}

If \(A\) is an \(n \times n\) matrix, then a non-zero vector
\(\symbf{x}\), in \(R^n\), is called an \ul{eigenvector} of \(A\) if
\(A\symbf{x}\) is a scalar multiple of \(\symbf{x}\); that is
\(A\symbf{x} = \lambda\symbf{x}\) for some scalar \(\lambda\). This
scalar \(\lambda\) is called an \ul{eigenvalue} of \(A\) and
\(\symbf{x}\) is said to be an \ul{eigenvector corresponding to}
\(\lambda\).

See, normally, multiplying a vector by a square matrix changes both the
magnitude and the direction of the vector. Really screws it up.

Some examples:

\(\begin{bmatrix}1 & 0 \\ 0 & 1\end{bmatrix}\begin{bmatrix}1 \\ 2\end{bmatrix}=\begin{bmatrix}1 \\ 2\end{bmatrix}\)

\(\begin{bmatrix}1 & 0 \\ 0 & 2 \end{bmatrix}\begin{bmatrix} 1 \\ 2 \end{bmatrix} = \begin{bmatrix}1 \\ 4\end{bmatrix}\)

\(\begin{bmatrix} 5 & 0 \\ 0 & 2 \end{bmatrix}\begin{bmatrix} 1 \\ 2 \end{bmatrix} = \begin{bmatrix} 5 \\ 4 \end{bmatrix}\)

\(\begin{bmatrix}0 & 1 \\ 1 & 0 \end{bmatrix}\begin{bmatrix}1 \\ 2 \end{bmatrix} = \begin{bmatrix}2 \\ 1 \end{bmatrix}\)

\(\begin{bmatrix}1 & 0 \\ 0 & 2\end{bmatrix}\begin{bmatrix}1 \\ 2\end{bmatrix} = \begin{bmatrix}1 \\ 4\end{bmatrix}\)

\(\begin{bmatrix}7 & 8 \\ -2 & 3 \end{bmatrix}\begin{bmatrix}1 \\ 2\end{bmatrix} = \begin{bmatrix}23 \\ 4\end{bmatrix}\)

However, there are some ways to get consistent results.

\subsubsection{Examples}\label{examples-2}

\paragraph{\texorpdfstring{\(\vec{x} = \begin{bmatrix}2 \\ 1 \end{bmatrix}\)
is an eigenvector of \(A = \begin{bmatrix}3 & -2 \\ 1 & 0\end{bmatrix}\)
because}{\textbackslash vec\{x\} = \textbackslash begin\{bmatrix\}2 \textbackslash\textbackslash{} 1 \textbackslash end\{bmatrix\} is an eigenvector of A = \textbackslash begin\{bmatrix\}3 \& -2 \textbackslash\textbackslash{} 1 \& 0\textbackslash end\{bmatrix\} because}}\label{vecx-beginbmatrix2-1-endbmatrix-is-an-eigenvector-of-a-beginbmatrix3--2-1-0endbmatrix-because}

\(A\vec{x} = \begin{bmatrix}3 & -2 \\ 1 & 0\end{bmatrix}\begin{bmatrix}2 \\ 1\end{bmatrix} = \begin{bmatrix}4 \\ 2\end{bmatrix} = 2\begin{bmatrix}2 \\ 1\end{bmatrix} = 2\vec{x} \therefore \lambda = 2\)

\paragraph{\texorpdfstring{Let
\(A = \begin{bmatrix} 1 & 6 \\ 5 & 2\end{bmatrix}, \vec{u} = \begin{bmatrix} 6 \\ -5\end{bmatrix}, \vec{v} = \begin{bmatrix}3 \\ -2 \end{bmatrix}\).
Are \(\vec{u}\) and \(\vec{v}\) eigenvectors of
\(A\)?}{Let A = \textbackslash begin\{bmatrix\} 1 \& 6 \textbackslash\textbackslash{} 5 \& 2\textbackslash end\{bmatrix\}, \textbackslash vec\{u\} = \textbackslash begin\{bmatrix\} 6 \textbackslash\textbackslash{} -5\textbackslash end\{bmatrix\}, \textbackslash vec\{v\} = \textbackslash begin\{bmatrix\}3 \textbackslash\textbackslash{} -2 \textbackslash end\{bmatrix\}. Are \textbackslash vec\{u\} and \textbackslash vec\{v\} eigenvectors of A?}}\label{let-a-beginbmatrix-1-6-5-2endbmatrix-vecu-beginbmatrix-6--5endbmatrix-vecv-beginbmatrix3--2-endbmatrix.-are-vecu-and-vecv-eigenvectors-of-a}

\(A\vec{u} = \begin{bmatrix} 1 & 6 \\ 5 & 2\end{bmatrix}\begin{bmatrix} 6 \\ -5\end{bmatrix} = \begin{bmatrix} 1(6)+6(-5) \\ 5(6)+2(-5)\end{bmatrix} = \begin{bmatrix} -24 \\ 20\end{bmatrix} = -4\begin{bmatrix}6 \\ -5\end{bmatrix} \therefore \ \lambda = -4\)

\(A\vec{v} = \begin{bmatrix} 1 & 6 \\ 5 & 2\end{bmatrix}\begin{bmatrix} 3 \\ -2\end{bmatrix} = \begin{bmatrix} 1(3)+6(-2) \\ 5(3)+2(-2)\end{bmatrix} = \begin{bmatrix} -9 \\ 11\end{bmatrix} \neq \lambda\vec{v}\)

\subsection{Eigenvector Homework Problem
(11/06)}\label{eigenvector-homework-problem-1106}

\textbf{Confirm by multiplication that \(\symbf{x}\) is an eigenvector
of \(\symbf{A}\), and find the corresponding eigenvalue.}

\subsubsection{\texorpdfstring{\(A = \begin{bmatrix}4 & 0 & 1 \\ 2 & 3 & 2 \\ 1 & 0 & 4 \end{bmatrix}; \symbf{x} = \begin{bmatrix}1 \\ 2 \\ 1\end{bmatrix}\)}{A = \textbackslash begin\{bmatrix\}4 \& 0 \& 1 \textbackslash\textbackslash{} 2 \& 3 \& 2 \textbackslash\textbackslash{} 1 \& 0 \& 4 \textbackslash end\{bmatrix\}; \textbackslash symbf\{x\} = \textbackslash begin\{bmatrix\}1 \textbackslash\textbackslash{} 2 \textbackslash\textbackslash{} 1\textbackslash end\{bmatrix\}}}\label{a-beginbmatrix4-0-1-2-3-2-1-0-4-endbmatrix-symbfx-beginbmatrix1-2-1endbmatrix}

\(A\symbf{x} = \begin{bmatrix}4 & 0 & 1 \\ 2 & 3 & 2 \\ 1 & 0 & 4 \end{bmatrix}\begin{bmatrix}1 \\ 2 \\ 1\end{bmatrix} = \begin{bmatrix}4(1)+0(2)+1(1) \\ 2(1)+3(2)+2(1) \\ 1(1)+0(2)+4(1)\end{bmatrix} = \begin{bmatrix}5 \\ 10 \\ 5\end{bmatrix} = 5\begin{bmatrix}1 \\ 2 \\ 1\end{bmatrix} \therefore \ \lambda = 5\)

\subsection{Finding Eigenvalues and Eigenvectors
(11/07)}\label{finding-eigenvalues-and-eigenvectors-1107}

Essential question:

\textbf{If we know an \(\symbf{n \times n}\) matrix \(\symbf{A}\), can
we find its \(\symbf{\lambda}\)?}

If \(A\vec{x} = \lambda\vec{x}\), then: \begin{align*}
A\vec{x} &= \lambda\vec{x} \\
A\vec{x} - \lambda\vec{x} &= \vec{0} \\
(A-\lambda I)\vec{x} &= \vec{0}
\end{align*} This equation is familiar. It's the homogeneous system of
equations \(A\vec{x} = \vec{0}\), the solution of which is the nullspace
of \(A-\lambda I\). Therefore, \(\vec{x}\) is an eigenvector of
\(A \iff \vec{x}\) is in the nullspace of \(A-\lambda I\).

In this situation, what do we know about that matrix?

Everything in the equivalent statements is false because \(\vec{x}\)
cannot be the zero vector. Therefore, we can see that
\(\det(A-\lambda I)\) OR \(\det(\lambda I-A)\) MUST be 0.

Big Idea: If \(A\) is an \(n \times n\) matrix, then \(\lambda\) is an
eigenvalue of \(A \iff \det(\lambda I-A) = 0\). This is called the
\ul{characteristic equation} of \(A\).

\subsubsection{\texorpdfstring{Find the characteristic equation and the
eigenvalues of
\(A=\begin{bmatrix}3 & 0 & 5\\ \frac{1}{5} & -1 & 0 \\ 1 & 1 & -2\end{bmatrix}\)}{Find the characteristic equation and the eigenvalues of A=\textbackslash begin\{bmatrix\}3 \& 0 \& 5\textbackslash\textbackslash{} \textbackslash frac\{1\}\{5\} \& -1 \& 0 \textbackslash\textbackslash{} 1 \& 1 \& -2\textbackslash end\{bmatrix\}}}\label{find-the-characteristic-equation-and-the-eigenvalues-of-abeginbmatrix3-0-5-frac15--1-0-1-1--2endbmatrix}

\begin{align*}
\det(\lambda I - A) &= 0 \\
\begin{vmatrix}\lambda-3 & 0 & 5 \\ -\frac{1}{5} & \lambda+1 & 0 \\ -1 & -1 & \lambda+2 \end{vmatrix} &= 0 \\
0 &= (\lambda-3)((\lambda+1)(\lambda+2))+5(\frac{1}{5} + \lambda+1) \\
0 &= (\lambda-3)(\lambda^2+3\lambda+2) \\
0 &= \lambda^3-2\lambda \\
0 &= \lambda(\lambda^2-2)
\lambda &= 0, \pm\sqrt{2}
\end{align*}

\subsubsection{\texorpdfstring{Find the characteristic equation and the
eigenvalues of
\(A = \begin{bmatrix}-1 & 0 & 1 \\ -1 & 3 & 0 \\ -4 & 13 & -1 \end{bmatrix}\)}{Find the characteristic equation and the eigenvalues of A = \textbackslash begin\{bmatrix\}-1 \& 0 \& 1 \textbackslash\textbackslash{} -1 \& 3 \& 0 \textbackslash\textbackslash{} -4 \& 13 \& -1 \textbackslash end\{bmatrix\}}}\label{find-the-characteristic-equation-and-the-eigenvalues-of-a-beginbmatrix-1-0-1--1-3-0--4-13--1-endbmatrix}

\begin{align*}
\begin{vmatrix}-1-\lambda & 0 & 1 \\ -1 & 3-\lambda & 0 \\ -4 & 13 & -\lambda \end{vmatrix} = 0 \\
(-1-\lambda) \left((3-\lambda)(-\lambda)-0(13)\right)+\left(-1(13)-(3-\lambda)(-4)\right) &= 0 \\
(-1-\lambda)(\lambda^2-3\lambda)+(-13-4\lambda+12) &= 0 \\
(-1-\lambda)(\lambda^2-3\lambda)+(-4\lambda-1) &= 0 \\
-\lambda^3+3\lambda^2+2 &= 0 \\
(-\lambda+2)(-\lambda^2-\lambda-1) &= 0 \\
(-\lambda+2)(-\lambda-1)(-\lambda+1) &= 0 \\
\lambda &= 2
\end{align*}

\newpage{}

\subsubsection{\texorpdfstring{Find the eigenvalues of
\(A = \begin{bmatrix}2 & 0 & 0 \\ 6 & 3 & 0 \\ 1 & 4 & 5 \end{bmatrix}\)}{Find the eigenvalues of A = \textbackslash begin\{bmatrix\}2 \& 0 \& 0 \textbackslash\textbackslash{} 6 \& 3 \& 0 \textbackslash\textbackslash{} 1 \& 4 \& 5 \textbackslash end\{bmatrix\}}}\label{find-the-eigenvalues-of-a-beginbmatrix2-0-0-6-3-0-1-4-5-endbmatrix}

\begin{align*}
\begin{vmatrix}\lambda-2 & 0 & 0 \\ 6 & \lambda-3 & 0 \\ 1 & 4 & \lambda-5\end{vmatrix} &= 0 \\
(\lambda-2)(\lambda-3)(\lambda-5) &= 0 \\
\lambda &= 2, 3, 5
\end{align*} Theorem 1: For a triangular matrix, the eigenvalues are the
elements on the main diagonal.

\subsubsection{\texorpdfstring{Find the eigenvalues of \(A^3\) if
\(A=\begin{bmatrix}\frac{1}{2} & 4 & 5 & -2 \\ 0 & -1 & 3 & -8 \\ 0 & 0 & 2 & 6 \\ 0 & 0 & 0 & 4\end{bmatrix}\)}{Find the eigenvalues of A\^{}3 if A=\textbackslash begin\{bmatrix\}\textbackslash frac\{1\}\{2\} \& 4 \& 5 \& -2 \textbackslash\textbackslash{} 0 \& -1 \& 3 \& -8 \textbackslash\textbackslash{} 0 \& 0 \& 2 \& 6 \textbackslash\textbackslash{} 0 \& 0 \& 0 \& 4\textbackslash end\{bmatrix\}}}\label{find-the-eigenvalues-of-a3-if-abeginbmatrixfrac12-4-5--2-0--1-3--8-0-0-2-6-0-0-0-4endbmatrix}

\begin{align*}
\lambda_A = \frac{1}{2}, -1, 2, 4 \\
\lambda_{A^3} = \frac{1}{8}, -1, 8, 64
\end{align*} Theorem 2: The eigenvalues of \(A^k\) are
\(\lambda_1^k, \lambda_2^k, ...\)

\subsubsection{\texorpdfstring{Give me a matrix with eigenvalues
\(\lambda = 0, 2, 5\)}{Give me a matrix with eigenvalues \textbackslash lambda = 0, 2, 5}}\label{give-me-a-matrix-with-eigenvalues-lambda-0-2-5}

\(A = \begin{bmatrix}0 & 0 & 0 \\ 1 & 2 & 0 \\ 2 & 2 & 5\end{bmatrix}\)

Theorem 3: A square matrix \(A\) is invertible \(\iff \lambda \neq 0\)
(which also means its determinant is 0).

\subsubsection{Finding eigenvectors!}\label{finding-eigenvectors}

Find the nontrivial eigenvectors of:

\(A = \begin{bmatrix}1 & 6 \\ 5 & 2\end{bmatrix}\) \begin{align*}
\begin{vmatrix}\lambda-1 & -6 \\ -5 & \lambda-2\end{vmatrix} &= 0 \\
(\lambda-1)(\lambda-2)-(-6)(-5) &= 0 \\
\lambda^2-3\lambda-28=0 \\
(\lambda-7)(\lambda+4) &= 0 \\
\lambda &= 7, -4
\end{align*} Substitute each \(\lambda\), one at a time into the
\(\lambda I - A\) matrix and find the null space.

For \(\lambda=-4\): \begin{align*}
\left(\begin{array}{cc|c}-5 & -6 & 0 \\ -5 & -6 & 0\end{array}\right) \\
\left(\begin{array}{cc|c}-5 & -6 & 0 \\ 0 & 0 & 0\end{array}\right) \\
\langle-\frac{6}{5}t, t\rangle \\
\vec{x} = \left\{\langle-6, 5\rangle\right\}
\end{align*} For \(\lambda=7\): \begin{align*}
\left(\begin{array}{cc|c}6 & -6 & 0 \\ -5 & 5 & 0\end{array}\right) \\
\left(\begin{array}{cc|c}6 & -6 & 0 \\ 0 & 0 & 0\end{array}\right) \\
\langle t, t\rangle \\
\vec{x} = \left\{\langle6, 6\rangle\right\}
\end{align*} Therefore, the eigen space is:
\(\left\{\langle-6, 5\rangle, \langle6, 6\rangle\right\}\)

\subsection{Diagonalization and Similar
Triangles}\label{diagonalization-and-similar-triangles}

Similar matrices: If \(A\) and \(D\) are square matrices, we say that A
and D are ``similar'' if there exists an invertible matrix \(P\) such
that:

\(D = P^{-1}AP\).

\subsubsection{Properties of Similar
Matricces}\label{properties-of-similar-matricces}

\begin{itemize}
\tightlist
\item
  They have the same determinant
\item
  If one is invertible, so is the other
\item
  They have the same trace
\item
  They have the same characteristic polynomial
\item
  They have the same eigenvalues
\end{itemize}

\subsubsection{Procedure}\label{procedure}

\begin{enumerate}
\def\labelenumi{\arabic{enumi}.}
\tightlist
\item
  Find the eigenvectors for the \(n \times n\) matrix \(A\).
\end{enumerate}

\begin{itemize}
\tightlist
\item
  Theorem: If an \(n \times n\) matrix \(A\) has \(n\) distinct
  eigenvalues, then \(A\) is \textbf{for sure} diagonalizable.
\end{itemize}

\begin{enumerate}
\def\labelenumi{\arabic{enumi}.}
\setcounter{enumi}{1}
\tightlist
\item
  Make matrix \(P\) out of the eigenveectors (\(P\) is the matrix that
  diagonalizes \(A\))
\item
  Check your work to find matrix \(D\) if reasonable
\end{enumerate}

\subsubsection{\texorpdfstring{Example: Find a matrix \(P\) that
diagonalizes \(A\) and compute
\(P^{-1}AP\)}{Example: Find a matrix P that diagonalizes A and compute P\^{}\{-1\}AP}}\label{example-find-a-matrix-p-that-diagonalizes-a-and-compute-p-1ap}

\begin{enumerate}
\def\labelenumi{\arabic{enumi}.}
\tightlist
\item
  \(A = \begin{bmatrix}3 & 7 \\ 5 & 5 \end{bmatrix}\)
\end{enumerate}

Find the eigenvalues: \begin{align*}
\begin{bmatrix}\lambda-3 & -7 \\ -5 & \lambda-5\end{bmatrix} &= 0 \\
(\lambda-3)(\lambda-5)-(-7)(-5) &= 0 \\
\lambda^2-5\lambda-3\lambda+15-35 &= 0 \\
\lambda^2-8\lambda-20 &= 0 \\
\lambda = -2, 10 
\end{align*}

Find the eigenvectors: \begin{align*}
\lambda = -2: \left[\begin{array}{cc|c}-5 & -7 & 0 \\ -5 & -7 & 0\end{array}\right] 
\vec{x} = \left\{\langle-7, 5\rangle\right\} \\
\lambda = 10: \left[\begin{array}{cc|c}-7 & -7 & 0 \\ -5 & 5 & 0\end{array}\right]
\vec{x} = \left\{\langle1,     1\rangle\right\}
\end{align*}

\newpage{}

Create the matrix \(P\):

\(P = \begin{bmatrix}-7 & 1 \\ 5 & 1\end{bmatrix}\)

Find matrix \(D\): \begin{align*}
D &= P^{-1}AP \\
&= \begin{bmatrix}7 & 1 \\ 5 & -1\end{bmatrix}^{-1}\begin{bmatrix}3 & 7 \\ 5 & 5\end{bmatrix}\begin{bmatrix}7 & 1 \\ 5 & -1\end{bmatrix} \\
&= \begin{bmatrix}-2 & 0 \\ 0 & 10\end{bmatrix}
\end{align*}

\subsubsection{Conclusion}\label{conclusion}

\begin{itemize}
\tightlist
\item
  If \(D\) has the same eigenvalues of \(A\) and if \(D\) must be
  diagonal, then \(D\) is \textbf{THE} diagonal matrix with eigenvalues
  of \(A\) on the diagonal.
\end{itemize}

\begin{enumerate}
\def\labelenumi{\arabic{enumi}.}
\setcounter{enumi}{1}
\tightlist
\item
  \(A = \begin{bmatrix}2 & 0 & -2 \\ 0 & 3 & 0 \\ 0 & 0 & 1\end{bmatrix}\)
\end{enumerate}

First, find \(D\): \begin{align*}
D = \begin{bmatrix} 2 & 0 & 0 \\ 0 & 3 & 0 \\ 0 & 0 & 1\end{bmatrix} \\
\end{align*}

Then, find \(P\): \begin{align*}
\lambda = 2: \left[\begin{array}{ccc|c}0 & 0 & 2 & 0 \\ 0 & -1 & 0 & 0 \\ 0 & 0 & 1 & 0\end{array}\right] \vec{x} = \langle1, 0, 0\rangle \\
\lambda = 3: \left[\begin{array}{ccc|c}1 & 0 & 2 & 0 \\ 0 & 0 & 0 & 0 \\ 0 & 0 & 2 & 0\end{array}\right] \vec{x} = \langle0, 1, 0\rangle \\
\lambda = 1: \left[\begin{array}{ccc|c}-1 & 0 & 2 & 0 \\ 0 & -2 & 0 & 0 \\ 0 & 0 & 0 & 0\end{array}\right] \vec{x} = \langle2, 0, 1\rangle \\
P = \begin{bmatrix}1 & 0 & 2 \\ 0 & 1 & 0 \\ 0 & 0 & 1\end{bmatrix}
\end{align*}

\subsection{More on Similar Matrices}\label{more-on-similar-matrices}

There are a few more properties of similar matrices:

\begin{itemize}
\tightlist
\item
  They have the same rank (non-zero eigenvalues)
\item
  They have the same nullity
\item
  They have the same column space
\item
  They have the same row space
\end{itemize}

\subsubsection{Example}\label{example-3}

**Matrix \(A\) is similar to the following matrix:

\(D = \begin{bmatrix}3 & -1 & 1 & 4 & 5 & 2 \\ 0 & -3 & 5 & -10 & -16 & 1 \\ 0 & 0 & 5 & 7 & -8 & 2 \\ 0 & 0 & 0 & 2 & 2 & 0 \\ 0 & 0 & 0 & 0 & 0 & 0 \\ 0 & 0 & 0 & 0 & 0 & 0\end{bmatrix}\)

Rank of \(A\): 4

Nullity of \(A\): 2

Eigenvalues: 3, -3, 5, 2, 0, 0

Characteristic Polynomial: \begin{align*}
\det(\lambda I - A) = 0 \\
\begin{vmatrix}\lambda-3 & 1 & -1 & -4 & -5 & -2 \\ 0 & \lambda+3 & -5 & 10 & 16 & -1 \\ 0 & 0 & \lambda-5 & -7 & 8 & -2 \\ 0 & 0 & 0 & \lambda-2 & -2 & 0 \\ 0 & 0 & 0 & 0 & \lambda & 0 \\ 0 & 0 & 0 & 0 & 0 & \lambda\end{vmatrix} = 0 \\
(\lambda-3)(\lambda+3)(\lambda-5)(\lambda-2)\lambda^2 = 0
\end{align*}

\newpage{}

\subsubsection{Some review}\label{some-review}

\begin{itemize}
\tightlist
\item
  \ul{Eigenspace} of \(\lambda\): The nullspace of \(\lambda I - A\).
  Each eigenvalue will have its own eigenspace.
\item
  \ul{Algebraic multiplicty}: The number of times a given \(\lambda\)
  appears as a root of the characteristic equation.
\item
  \ul{Geometric multiplicity}: The number of eigenvectors it maps to.
\end{itemize}

\paragraph{Theorem: Geometric and Algebraic
Multiplicity}\label{theorem-geometric-and-algebraic-multiplicity}

If \(A\) is a square matrix, then: a. For every eigenvalue of \(A\), the
geometric multiplicity is less than or equal to the algebraic
multiplicity. b. \(A\) is diagonalizable \(\iff\) the geometric
multiplicity of each eigenvalue is equal to the algebraic multiplicity.

\subsection{Similar Matrices Continued
(11/13/2023)}\label{similar-matrices-continued-11132023}

\subsubsection{Warm-Up}\label{warm-up}

Can you write a new statement involving eigenvalues to add to the list
of equivalent statements?

\(\lambda = 0\) is not an eigenvalue of \(A\)

\subsubsection{Homework Review}\label{homework-review}

\(A = \begin{bmatrix} 19 & -9 & -6 \\ 25 & -11 & -9 \\ 17 & -9 & -4 \end{bmatrix}\)
\begin{align*}
\begin{bmatrix}\lambda-19 & 9 & 6 \\ -25 & \lambda+11 & 9 \\ 17 & 9 & \lambda + 4\end{bmatrix}
&= (\lambda-19)((\lambda+11)(\lambda+4)-81) + 25(9\lambda+36-54) - 17(81-6\lambda-66) \\
&= (\lambda-1)^2(\lambda-2)
\end{align*}

\(\lambda=1\) has an algebraic multiplicity of 2 and \(\lambda=2\) has
an algebraic multiplicity of 1.

\subsubsection{\texorpdfstring{Suppose that a characteristic polynomial
of some matrix \(A\) is found to
be:}{Suppose that a characteristic polynomial of some matrix A is found to be:}}\label{suppose-that-a-characteristic-polynomial-of-some-matrix-a-is-found-to-be}

\(p(\lambda) = (\lambda-1)(\lambda-3)^2(\lambda=4)^3\)

\begin{enumerate}
\def\labelenumi{\alph{enumi}.}
\tightlist
\item
  What are the dimensions of \(A\)?
\end{enumerate}

\(6\times6\)

\begin{enumerate}
\def\labelenumi{\alph{enumi}.}
\setcounter{enumi}{1}
\tightlist
\item
  What are the algebraic multiplicities of each eigenvalue?
\end{enumerate}

\(\lambda=1: 1, \lambda=3:2, \lambda=4:3\)

\begin{enumerate}
\def\labelenumi{\alph{enumi}.}
\setcounter{enumi}{2}
\tightlist
\item
  What are the possible dimensions of the eigenspace associated with
  each of the eigenvalues?
\end{enumerate}

\(\lambda=1:1, \lambda=3:1 \text{ or } 2, \lambda=4:1 \text{ or } 2 \text{ or } 3\)

\begin{enumerate}
\def\labelenumi{\alph{enumi}.}
\setcounter{enumi}{3}
\tightlist
\item
  If \(\{v_1, v_2\}\) is a linearly independent set of eigenvectors of
  \(A\), all of which correspond to the same eigenvalue of \(A\), what
  can you say about the eigenvalue?
\end{enumerate}

The eigenvalue must be 3 or 4.

\section{Semester II}\label{semester-ii}

\subsection{Introduction to Multivariable functions
(01/04)}\label{introduction-to-multivariable-functions-0104}

\subsubsection{Definition of a Multivariable
Function}\label{definition-of-a-multivariable-function}

Suppose \(D\) is a set of \(n\)-tuples of real numbers
(\(x_1, x_2, ..., x_n\)). A function \(f\) on \(D\) is a rule that
assigns a real number

\(w = f(x_1, x_2, ..., x_n)\)

to each element in \(D\). The set \(D\) is the function's
\textbf{domain}. The set of \(w\)-values taken on by \(f\) is the
function's \textbf{range}. The symbol \(w\) is the \textbf{dependent
variable} of \(f\), and \(f\) is a function of the \(n\)
\textbf{independent variables} \(x_1\) to \(x_n\). The \(x\)'s are the
function's \textbf{input variables}; \(w\) is the function's
\textbf{output variable}.

\subsection{Limits and Continuity
(01/05)}\label{limits-and-continuity-0105}

\subsubsection{Level Curve, Graph,
Surface}\label{level-curve-graph-surface}

The set of points in the plane where a function \(f(x, y)\) has a
constant value \(k\) is called a \textbf{level curve} of \(f\). The
graph of \(f\) is the set of all points \((x, y, z)\) in space, where
\(z = f(x, y)\). The graph of \(f\) is a surface in space.

\subsubsection{Limits With Multivariable
Functions}\label{limits-with-multivariable-functions}

\emph{Limits}: Let \(f\) be a function of two variables defined on an
open region, except possibly at \$(x\_0, \(y_0\)).

In 2-D:

\(\lim_{x\rightarrow c} f(x)\) exists iff
\(\lim_{x\rightarrow c^-} f(x) = \lim_{x\rightarrow c^+} f(x) = L \in \mathbb{R}\)

In 3-D:

\(\lim_{(x, y) \rightarrow (x_0, y_0)} f(x, y)\) exists iff
\(\lim_{(x, y) \rightarrow (x_0, y_0)^-} f(x, y) = L \in \mathbb{R}\)
from \ul{all} directions.

\subsubsection{Examples}\label{examples-3}

\begin{enumerate}
\def\labelenumi{\arabic{enumi}.}
\item
  \(\lim_{(x, y) \rightarrow (1, 2)} \cfrac{5x^2y}{x^2+y^2} = \cfrac{5 \cdot 1 \cdot 2}{1 + 4} = \cfrac{10}{5} = 2\)
\item
  \(\lim_{(x, y) \rightarrow (1, 1)} \cfrac{x-y}{x^2-y^2} = \cfrac{0}{0}\)
\end{enumerate}

Since this is indeterminate, you need to switch to traditional limit
solving methods. There is no \emph{L'Hopital's Rule} for multivariable
functions. Instead, factor the numerator and denominator and cancel out
common factors.

\(\lim_{(x, y) \rightarrow (1, 1)} \cfrac{x-y}{(x-y)(x+y)} = \cfrac{1}{2}\)

\begin{enumerate}
\def\labelenumi{\arabic{enumi}.}
\setcounter{enumi}{2}
\tightlist
\item
  \(\lim_{(x, y) \rightarrow (4, 3)} \cfrac{\sqrt{x}-\sqrt{y+1}}{x-y-1}\)
\end{enumerate}

\subsubsection{Homework}\label{homework}

\paragraph{\texorpdfstring{Evaluate the limit
\(\lim_{(x, y) \rightarrow (0, 0)} \cfrac{3x^2-y^2+5}{x^2+y^2+2}\)}{Evaluate the limit \textbackslash lim\_\{(x, y) \textbackslash rightarrow (0, 0)\} \textbackslash cfrac\{3x\^{}2-y\^{}2+5\}\{x\^{}2+y\^{}2+2\}}}\label{evaluate-the-limit-lim_x-y-rightarrow-0-0-cfrac3x2-y25x2y22}

\(\left.\lim_{(x, y) \rightarrow (0, 0)} \cfrac{3x^2-y^2+5}{x^2+y^2+2}\right\vert_{(0,0)} = \cfrac{5}{2}\)

\paragraph{\texorpdfstring{Evaluate the limit
\(\lim_{(x, y) \rightarrow (1, 1)} \cos\sqrt[3]{|xy|-1}\)}{Evaluate the limit \textbackslash lim\_\{(x, y) \textbackslash rightarrow (1, 1)\} \textbackslash cos\textbackslash sqrt{[}3{]}\{\textbar xy\textbar-1\}}}\label{evaluate-the-limit-lim_x-y-rightarrow-1-1-cossqrt3xy-1}

\(\left.\lim_{(x, y) \rightarrow (1, 1)} \cos\sqrt[3]{|xy|-1}\right\vert_{(1, 1} = \cos0 = 1\)

\paragraph{\texorpdfstring{Evaluate the limit
\(\lim_{(x, y) \rightarrow (1, 1)} \cfrac{xy-y-2x+2}{x-1}\)}{Evaluate the limit \textbackslash lim\_\{(x, y) \textbackslash rightarrow (1, 1)\} \textbackslash cfrac\{xy-y-2x+2\}\{x-1\}}}\label{evaluate-the-limit-lim_x-y-rightarrow-1-1-cfracxy-y-2x2x-1}

\(\lim_{(x, y) \rightarrow (1, 1)} \cfrac{xy-y-2x+2}{x-1}=\cfrac{(x-1)(y-2)}{x-1}=\left.y-2\right\vert_{(1,1)} = -1\)

\paragraph{\texorpdfstring{On what interval is the function
\(f(x, y) = \sin(x+y)\)
continuous?}{On what interval is the function f(x, y) = \textbackslash sin(x+y) continuous?}}\label{on-what-interval-is-the-function-fx-y-sinxy-continuous}

The \(\sin\) function is continuous everywhere, so
\((x, y) \in \mathbb{R}^2\)

\paragraph{\texorpdfstring{On what interval is the function
\(f(x, y, z) = x^2+y^2-2z^2\)
continuous?}{On what interval is the function f(x, y, z) = x\^{}2+y\^{}2-2z\^{}2 continuous?}}\label{on-what-interval-is-the-function-fx-y-z-x2y2-2z2-continuous}

\((x, y, z) \in \mathbb{R}^3\)

\paragraph{\texorpdfstring{On what interval is the function
\(f(x, y, z) = xy\sin\left(\frac{1}{z}\right)\)
continuous?}{On what interval is the function f(x, y, z) = xy\textbackslash sin\textbackslash left(\textbackslash frac\{1\}\{z\}\textbackslash right) continuous?}}\label{on-what-interval-is-the-function-fx-y-z-xysinleftfrac1zright-continuous}

Can't divide by zero, so \((x, y, z) \in \mathbb{R}^3 | z \ne 0\)

\subsection{Limits that DO NOT EXIST in 3-Space
(01/08)}\label{limits-that-do-not-exist-in-3-space-0108}

\subsubsection{\texorpdfstring{Question: Does the function
\(f(x,y) = \cfrac{2x^2y}{x^4+y^2}\) have a limit as (x,y) approaches
(0,0)?}{Question: Does the function f(x,y) = \textbackslash cfrac\{2x\^{}2y\}\{x\^{}4+y\^{}2\} have a limit as (x,y) approaches (0,0)?}}\label{question-does-the-function-fxy-cfrac2x2yx4y2-have-a-limit-as-xy-approaches-00}

Nope, it approaches the point sp.differently.

\subsubsection{\texorpdfstring{Find \(\left.f(x, y)\right\vert_{y=x^2}\)
and compute \(\lim_{(x, y) \rightarrow (0, 0)} f(x, y)\) along
\(y=x^2\)}{Find \textbackslash left.f(x, y)\textbackslash right\textbackslash vert\_\{y=x\^{}2\} and compute \textbackslash lim\_\{(x, y) \textbackslash rightarrow (0, 0)\} f(x, y) along y=x\^{}2}}\label{find-left.fx-yrightvert_yx2-and-compute-lim_x-y-rightarrow-0-0-fx-y-along-yx2}

\(\left.f(x,y) = \cfrac{2x^2y}{x^4 +y^2}\right\vert_{y=x^2} = \cfrac{2x^4}{2x^4} = 1\)

\subsubsection{\texorpdfstring{Find
\(\left.f(x, y)\right\vert_{y=-x^2}\) and compute
\(\lim_{(x, y) \rightarrow (0, 0)} f(x, y)\) along
\(y=-x^2\)}{Find \textbackslash left.f(x, y)\textbackslash right\textbackslash vert\_\{y=-x\^{}2\} and compute \textbackslash lim\_\{(x, y) \textbackslash rightarrow (0, 0)\} f(x, y) along y=-x\^{}2}}\label{find-left.fx-yrightvert_y-x2-and-compute-lim_x-y-rightarrow-0-0-fx-y-along-y-x2}

\(\left.f(x,y) = \cfrac{2x^2y}{x^4 +y^2}\right\vert_{y=x^2} = \cfrac{-2x^4}{2x^4} = 1\)

\subsubsection{\texorpdfstring{Explain why we can conclude that
\(\lim_{(x, y) \rightarrow (0, 0)} f(x,y)\) does not
exist}{Explain why we can conclude that \textbackslash lim\_\{(x, y) \textbackslash rightarrow (0, 0)\} f(x,y) does not exist}}\label{explain-why-we-can-conclude-that-lim_x-y-rightarrow-0-0-fxy-does-not-exist}

\(\lim_{(x,y) \rightarrow(0,0)}f(x,y)\) along \(y=x^2\)
\(\ne \lim_{(x, y)\rightarrow(0, 0)}f(x, y)\) along \(y=-x^2\)

\subsubsection{\texorpdfstring{How did we know to choose \(y=x^2\) and
\(y=-x^2\) to evaluate the
limit?}{How did we know to choose y=x\^{}2 and y=-x\^{}2 to evaluate the limit?}}\label{how-did-we-know-to-choose-yx2-and-y-x2-to-evaluate-the-limit}

You want to try to choose things that, as you plug them in, you get a
nice expression that you can simplify. For this problem in particular,
we chose paths towards \((0, 0)\) that would be easy to solve and yield
sp.different answers upon simplification. We could've also used the
equation of the x-axis (\(y=0\)).

\subsubsection{\texorpdfstring{Show that these functions have no limit
as \((x, y)\) approaches \((0, 0)\) by considering sp.different paths of
approach.}{Show that these functions have no limit as (x, y) approaches (0, 0) by considering sp.different paths of approach.}}\label{show-that-these-functions-have-no-limit-as-x-y-approaches-0-0-by-considering-sp.different-paths-of-approach.}

\paragraph{\texorpdfstring{\(f(x,y) = \cfrac{x^4}{x^4+y^2}\)}{f(x,y) = \textbackslash cfrac\{x\^{}4\}\{x\^{}4+y\^{}2\}}}\label{fxy-cfracx4x4y2}

\(\lim_{(x, y)\rightarrow(0, 0)} f(x, y)\) along
\(y=2x^2 = \cfrac{x^4}{5x^4}=\cfrac{1}{5}\)
\(\ne \lim_{(x, y)\rightarrow(0, 0)} f(x, y)\) along
\(y=x^2 = \cfrac{x^4}{2x^4} = \cfrac{1}{2}\)

\paragraph{\texorpdfstring{\(f(x, y) = \cfrac{x^2+y}{y}\)}{f(x, y) = \textbackslash cfrac\{x\^{}2+y\}\{y\}}}\label{fx-y-cfracx2yy}

\(\lim_{(x, y)\rightarrow(0,0)} f(x, y)\) along
\(y=x^2 = \cfrac{2y}{y} = 2\)
\(\ne \lim_{(x, y)\rightarrow(0,0)} f(x, y)\) along
\(y=2x^2 = \cfrac{3x^2}{2x^2} = \cfrac{3}{2}\)

\subsection{Partial Derivatives (01/10)}\label{partial-derivatives-0110}

\subsubsection{First Order Partial
Derivatives}\label{first-order-partial-derivatives}

A \textbf{partial derivative} is obtained by holding all but one of the
independent variables constant and differentiating with respect to that
variable.

\paragraph{Notation}\label{notation}

\(\frac{\partial f}{\partial x} = f_x = \frac{\partial}{\partial x}f(x, y)\)
is the partial derivative of \(f\) with respect to \(x\)

\subsubsection{Examples}\label{examples-4}

\paragraph{\texorpdfstring{\(f(x, y) = 2x^3y-4x^2y^3+5x^4\)}{f(x, y) = 2x\^{}3y-4x\^{}2y\^{}3+5x\^{}4}}\label{fx-y-2x3y-4x2y35x4}

\(f_x = 6x^2y-8xy^3+20x^3\)\\
\(f_y = 2x^3-12x^2y^2\)

\paragraph{\texorpdfstring{\(f(x, y) = 4x^2y-8x^3y^4+2xy^7\)}{f(x, y) = 4x\^{}2y-8x\^{}3y\^{}4+2xy\^{}7}}\label{fx-y-4x2y-8x3y42xy7}

\(f_x = 8xy-24x^2y^4+2y^7\)\\
\(f_y = 4x^2-32x^3y^3+14xy^6\)

\paragraph{\texorpdfstring{\(f(x,y)=\tan(2x-y)\)}{f(x,y)=\textbackslash tan(2x-y)}}\label{fxytan2x-y}

\(f_x = 2\sec^2(2x-y)\)\\
\(f_y = -\sec^2(2x-y)\)

\subsubsection{The Second Fundamental Theorem of (Multivariable)
Calculus}\label{the-second-fundamental-theorem-of-multivariable-calculus}

\begin{itemize}
\tightlist
\item
  \(\frac{\mathrm{d}}{\mathrm{d}x}\int_5^x f(t) \mathrm{d}t = f(x)\)
\item
  \(\frac{\mathrm{d}}{\mathrm{d}x}\int_5^{x^2} f(t) \mathrm{d}t = 2x \cdot f(x^2)\)
\end{itemize}

\subsubsection{Examples}\label{examples-5}

\paragraph{\texorpdfstring{\(f(x,y) = \int_{3x}^{2y}(t^2-1)\mathrm{d}t\)}{f(x,y) = \textbackslash int\_\{3x\}\^{}\{2y\}(t\^{}2-1)\textbackslash mathrm\{d\}t}}\label{fxy-int_3x2yt2-1mathrmdt}

\begin{Shaded}
\begin{Highlighting}[numbers=left,,]
\CommentTok{\# import libraries}
\ImportTok{import}\NormalTok{ sympy }\ImportTok{as}\NormalTok{ sp}
\ImportTok{from}\NormalTok{ IPython.display }\ImportTok{import}\NormalTok{ display, Math, Latex}
\NormalTok{x, y, t, z, r, θ, p, q }\OperatorTok{=}\NormalTok{ sp.symbols(}\StringTok{\textquotesingle{}x y t z r θ p q\textquotesingle{}}\NormalTok{, real}\OperatorTok{=}\VariableTok{True}\NormalTok{)}
\end{Highlighting}
\end{Shaded}

\begin{Shaded}
\begin{Highlighting}[numbers=left,,]
\NormalTok{f }\OperatorTok{=}\NormalTok{ sp.integrate(t}\OperatorTok{**}\DecValTok{2}\OperatorTok{{-}}\DecValTok{1}\NormalTok{, (t, }\DecValTok{3}\OperatorTok{*}\NormalTok{x, }\DecValTok{2}\OperatorTok{*}\NormalTok{y))}
\NormalTok{display(sp.diff(f, x))}
\NormalTok{display(sp.diff(f, y))}
\end{Highlighting}
\end{Shaded}

$\displaystyle 3 - 27 x^{2}$

$\displaystyle 8 y^{2} - 2$

\paragraph{\texorpdfstring{Find both partial derivatives and evaluate
each at the point
\((1, \ln 2)\)}{Find both partial derivatives and evaluate each at the point (1, \textbackslash ln 2)}}\label{find-both-partial-derivatives-and-evaluate-each-at-the-point-1-ln-2}

\begin{enumerate}
\def\labelenumi{\arabic{enumi}.}
\tightlist
\item
  \(f(x, y) = xe^{x^2y}\)
\end{enumerate}

\begin{Shaded}
\begin{Highlighting}[numbers=left,,]
\NormalTok{f }\OperatorTok{=}\NormalTok{ x}\OperatorTok{*}\NormalTok{sp.exp(x}\OperatorTok{**}\DecValTok{2}\OperatorTok{*}\NormalTok{y)}
\NormalTok{display(sp.diff(f, x))}
\NormalTok{display(sp.diff(f, y))}
\NormalTok{display(sp.diff(f, x).subs(\{x:}\DecValTok{1}\NormalTok{, y:sp.ln(}\DecValTok{2}\NormalTok{)\}))}
\NormalTok{display(sp.diff(f, y).subs(\{x:}\DecValTok{1}\NormalTok{, y:sp.ln(}\DecValTok{2}\NormalTok{)\}))}
\end{Highlighting}
\end{Shaded}

$\displaystyle 2 x^{2} y e^{x^{2} y} + e^{x^{2} y}$

$\displaystyle x^{3} e^{x^{2} y}$

$\displaystyle 2 + 4 \log{\left(2 \right)}$

$\displaystyle 2$

\subsection{2nd Partial Derivatives
(01/11)}\label{nd-partial-derivatives-0111}

\subsubsection{\texorpdfstring{First, let's see this one. If
\(f(x, y, z) = x\sin(y+3z)\), find
\(f_x, f_y, f_z\).}{First, let's see this one. If f(x, y, z) = x\textbackslash sin(y+3z), find f\_x, f\_y, f\_z.}}\label{first-lets-see-this-one.-if-fx-y-z-xsiny3z-find-f_x-f_y-f_z.}

\begin{Shaded}
\begin{Highlighting}[numbers=left,,]
\NormalTok{f }\OperatorTok{=}\NormalTok{ x}\OperatorTok{*}\NormalTok{sp.sin(y}\OperatorTok{+}\DecValTok{3}\OperatorTok{*}\NormalTok{z)}
\NormalTok{display(sp.diff(f, x))}
\NormalTok{display(sp.diff(f, y))}
\NormalTok{display(sp.diff(f, z))}
\end{Highlighting}
\end{Shaded}

$\displaystyle \sin{\left(y + 3 z \right)}$

$\displaystyle x \cos{\left(y + 3 z \right)}$

$\displaystyle 3 x \cos{\left(y + 3 z \right)}$

\subsubsection{2nd-Order Partial
Derivatives}\label{nd-order-partial-derivatives}

The \textbf{second-order partial derivatives} of \(f\) are the partial
derivatives of the partial derivatives of \(f\).

\paragraph{Notation}\label{notation-1}

\(f_{xx} = \frac{\partial^2 f}{\partial x^2} = \frac{\partial}{\partial x}\left(\frac{\partial f}{\partial x}\right)\)

\subsubsection{Mixed Partial
Derivatives}\label{mixed-partial-derivatives}

The \textbf{mixed partial derivatives} of \(f\) are the partial
derivatives of \(f\) with respect to many other different variables.

\paragraph{Notation}\label{notation-2}

\(f_{xy} = \frac{\partial^2 f}{\partial x \partial y} = \frac{\partial}{\partial x}\left(\frac{\partial f}{\partial y}\right)\)

To compute a mixed partial derivative, you must evaluate the order from
right to left. For example, \(f_{xy}\) is the partial derivative of
\(f_x\) with respect to \(y\), which is notated as
\(\frac{\partial^2 f}{\partial x \partial y}\).

\subsubsection{Theorem (Clairaut's
Theorem)}\label{theorem-clairauts-theorem}

If \(f_{xy}\) and \(f_{yx}\) are continuous on an open region \(D\),
then \(f_{xy}(a, b) = f_{yx}(a, b)\) on \(D\).

\subsubsection{Examples}\label{examples-6}

\paragraph{\texorpdfstring{Given that \(f(x, y) = x^2y-y^3+\ln x\), find
\textbf{all} 2nd order partial
derivatives}{Given that f(x, y) = x\^{}2y-y\^{}3+\textbackslash ln x, find all 2nd order partial derivatives}}\label{given-that-fx-y-x2y-y3ln-x-find-all-2nd-order-partial-derivatives}

f = x**2*y - y**3 + sp.ln(x) display(sp.diff(f, y, x))
display(sp.diff(f, x, y)) display(sp.diff(f, x, x)) display(sp.diff(f,
y, y))

\begin{verbatim}


#### Given that $w = xy+\frac{e^y}{y^2+1}$, find $\frac{\partial^2 w}{\partial x \partial y}$ and $\frac{\partial^2 w}{\partial y \partial x}$

For this problem, you need to "choose wisely." Knowing that the function and it's partial derivatives are continuous on the open region $\mathbb{R}^2$, we can choose $w_{xy} = 1$.

### Partial Derivatives Homework (01/11)

### Find $\frac{\partial f}{\partial x}$ and $\frac{\partial f}{\partial y}$ for $f(x, y)$

#### $f(x, y) = x(y-1)$

```{python}
f = x*(y-1)
display(sp.diff(f, x))
display(sp.diff(f, y))
\end{verbatim}

\subsubsection{\texorpdfstring{\(f(x, y) = 5xy - 7x^2 - y^2 + 3x - 6y +2\)}{f(x, y) = 5xy - 7x\^{}2 - y\^{}2 + 3x - 6y +2}}\label{fx-y-5xy---7x2---y2-3x---6y-2}

\begin{Shaded}
\begin{Highlighting}[numbers=left,,]
\NormalTok{f = 5*x*y {-} 7*x**2 {-} y**2 + 3*x {-} 6*y + 2}
\NormalTok{display(sp.diff(f, x))}
\NormalTok{display(sp.diff(f, y))}
\end{Highlighting}
\end{Shaded}

\subsubsection{\texorpdfstring{\(f(x, y) = \sqrt{x^2+y^2}\)}{f(x, y) = \textbackslash sqrt\{x\^{}2+y\^{}2\}}}\label{fx-y-sqrtx2y2}

\begin{Shaded}
\begin{Highlighting}[numbers=left,,]
\NormalTok{f = sp.sqrt(x**2 + y**2)}
\NormalTok{display(sp.diff(f, x))}
\NormalTok{display(sp.diff(f, y))}
\end{Highlighting}
\end{Shaded}

\subsubsection{\texorpdfstring{\(f(x, y) = e^{(x+y+1)}\)}{f(x, y) = e\^{}\{(x+y+1)\}}}\label{fx-y-exy1}

\begin{Shaded}
\begin{Highlighting}[numbers=left,,]
\NormalTok{f = sp.exp(x+y+1) }
\NormalTok{display(sp.diff(f, x))}
\NormalTok{display(sp.diff(f, y))}
\end{Highlighting}
\end{Shaded}

\subsubsection{\texorpdfstring{\(f(x, y) = \int_x^y g(t) \mathrm{d}t\)}{f(x, y) = \textbackslash int\_x\^{}y g(t) \textbackslash mathrm\{d\}t}}\label{fx-y-int_xy-gt-mathrmdt}

\begin{Shaded}
\begin{Highlighting}[numbers=left,,]
\NormalTok{g = sp.Function(\textquotesingle{}g\textquotesingle{})}
\NormalTok{f = sp.integrate(g(t), (t, x, y))}
\NormalTok{display(sp.diff(f, x))}
\NormalTok{display(sp.diff(f, y))}
\end{Highlighting}
\end{Shaded}

\subsubsection{\texorpdfstring{\(f(x, y, z) = xy+yz+xz\)}{f(x, y, z) = xy+yz+xz}}\label{fx-y-z-xyyzxz}

\begin{Shaded}
\begin{Highlighting}[numbers=left,,]
\NormalTok{f = x*y + y*z + x*z}
\NormalTok{display(sp.diff(f, x))}
\NormalTok{display(sp.diff(f, y))}
\NormalTok{display(sp.diff(f, z))}
\end{Highlighting}
\end{Shaded}

\subsubsection{\texorpdfstring{\(f(x, y, z) = \ln(x + 2y + 3z)\)}{f(x, y, z) = \textbackslash ln(x + 2y + 3z)}}\label{fx-y-z-lnx-2y-3z}

\begin{Shaded}
\begin{Highlighting}[numbers=left,,]
\NormalTok{f = sp.ln(x + 2*y + 3*z)}
\NormalTok{display(sp.diff(f, x))}
\NormalTok{display(sp.diff(f, y))}
\NormalTok{display(sp.diff(f, z))}
\end{Highlighting}
\end{Shaded}

\subsubsection{\texorpdfstring{If \(f(x,y) = x\cos(y)+ye^x\), find
\(\frac{\partial^2 f}{\partial x^2}, \frac{\partial^2 f}{\partial y^2}, \frac{\partial^2 f}{\partial x \partial y},\)
and
\(\frac{\partial^2 f}{\partial y \partial x}\)}{If f(x,y) = x\textbackslash cos(y)+ye\^{}x, find \textbackslash frac\{\textbackslash partial\^{}2 f\}\{\textbackslash partial x\^{}2\}, \textbackslash frac\{\textbackslash partial\^{}2 f\}\{\textbackslash partial y\^{}2\}, \textbackslash frac\{\textbackslash partial\^{}2 f\}\{\textbackslash partial x \textbackslash partial y\}, and \textbackslash frac\{\textbackslash partial\^{}2 f\}\{\textbackslash partial y \textbackslash partial x\}}}\label{if-fxy-xcosyyex-find-fracpartial2-fpartial-x2-fracpartial2-fpartial-y2-fracpartial2-fpartial-x-partial-y-and-fracpartial2-fpartial-y-partial-x}

\begin{Shaded}
\begin{Highlighting}[numbers=left,,]
\NormalTok{f = x*sp.cos(y) + y*sp.exp(x)}
\NormalTok{display(sp.diff(f, x, x))}
\NormalTok{display(sp.diff(f, y, y))}
\NormalTok{display(sp.diff(f, x, y))}
\NormalTok{display(sp.diff(f, y, x))}
\end{Highlighting}
\end{Shaded}

\subsection{Definition of Differentiability
(01/16)}\label{definition-of-differentiability-0116}

So, how do we solve the apparent contradiction?

Defintion of differentiability in multivaraible calculus:

The function \(f(x, y)\) is said to be differentiable if there exists a
\emph{linear function} at the point (a, b):\\
\(L(x, y) = f(a, b) + p(x-a) + q(y-b)\)

\subsubsection{Remember local
linearization?}\label{remember-local-linearization}

\ldots what it means for a function to be ``locally linear?''

Consider the following:

Is this function differentiable at \(x=0\)?

\(f(x) = |x| + 1\)

What about this one?

\(g(x) = \sqrt{x^2+0.0001} + 0.99\)

The linearization of a function \(f(x,y)\) at a point \((x_0, y_0)\)
where \(f\) is differentiable is given by the function:

\begin{align*}
L(x, y) = f(x_0,y_0)+f_x(x_0,y_0)(x-x_0)+f_y(x_0,y_0)(y-y_0) \
z = z_0 + m_x(x-x_0)+m_y(y-y_0) \ 
0 = m_x(x-x_0)+m_y(y-y_0)+(z-z_0) \
0 = A(x-x_0) + B(y-y) + C(z-z_0)
\end{align*}x(x-x\_0)+m\_y(y-y\_0) ~ 0 =
m\_x(x-x\_0)+m\_y(y-y\_0)+(z-z\_0)\\
0 = A(x-x\_0) + B(y-y) + C(z-z\_0) \textbackslash end\{align*\}
\ldots where \(\vec{n} = \langle A, B, C \rangle\) and
\(p = (x_0, y_0, z_0)\)

\subsubsection{Examples}\label{examples-7}

\paragraph{\texorpdfstring{Find the linearization of
\(f(x, y) = y^2 + 2xy - \frac{1}{2}x^2\) at the point
\((2, 3)\).}{Find the linearization of f(x, y) = y\^{}2 + 2xy - \textbackslash frac\{1\}\{2\}x\^{}2 at the point (2, 3).}}\label{find-the-linearization-of-fx-y-y2-2xy---frac12x2-at-the-point-2-3.}

\subsubsection{Homework}\label{homework-1}

\paragraph{\texorpdfstring{Find the linearization of
\(f(x, y) = x^2 + y^2 + 1\) at
\((1, 1)\)}{Find the linearization of f(x, y) = x\^{}2 + y\^{}2 + 1 at (1, 1)}}\label{find-the-linearization-of-fx-y-x2-y2-1-at-1-1}

\begin{Shaded}
\begin{Highlighting}[numbers=left,,]
\NormalTok{f = x**2 + y**2 + 1}
\NormalTok{x\_0 = 1}
\NormalTok{y\_0 = 1}
\NormalTok{z\_0 = f.subs(\{x:x\_0, y:y\_0\})}
\NormalTok{f\_x = sp.diff(f, x).subs(\{x:x\_0, y:y\_0\})}
\NormalTok{f\_y = sp.diff(f, y).subs(\{x:x\_0, y:y\_0\})}
\NormalTok{display(z\_0 + f\_x*(x{-}x\_0) + f\_y*(y{-}y\_0))}
\end{Highlighting}
\end{Shaded}

\paragraph{\texorpdfstring{Find the linearization of
\(f(x, y) = 3x-4y+5\) at
\((1, 1)\)}{Find the linearization of f(x, y) = 3x-4y+5 at (1, 1)}}\label{find-the-linearization-of-fx-y-3x-4y5-at-1-1}

\begin{Shaded}
\begin{Highlighting}[numbers=left,,]
\NormalTok{f = 3*x {-} 4*y + 5}
\NormalTok{x\_0 = 1}
\NormalTok{y\_0 = 1}
\NormalTok{z\_0 = f.subs(\{x:x\_0, y:y\_0\})}
\NormalTok{f\_x = sp.diff(f, x).subs(\{x:x\_0, y:y\_0\})}
\NormalTok{f\_y = sp.diff(f, y).subs(\{x:x\_0, y:y\_0\})}
\NormalTok{display(z\_0 + f\_x*(x{-}x\_0) + f\_y*(y{-}y\_0))}
\end{Highlighting}
\end{Shaded}

\paragraph{\texorpdfstring{Find the linearization of
\(f(x, y) = e^x \cos y\) at
\((0, \frac{\pi}{2})\)}{Find the linearization of f(x, y) = e\^{}x \textbackslash cos y at (0, \textbackslash frac\{\textbackslash pi\}\{2\})}}\label{find-the-linearization-of-fx-y-ex-cos-y-at-0-fracpi2}

\begin{Shaded}
\begin{Highlighting}[numbers=left,,]
\NormalTok{f = sp.exp(x)*sp.cos(y)}
\NormalTok{x\_0 = 0}
\NormalTok{y\_0 = sp.pi/2}
\NormalTok{z\_0 = f.subs(\{x:x\_0, y:y\_0\})}
\NormalTok{f\_x = sp.diff(f, x).subs(\{x:x\_0, y:y\_0\})}
\NormalTok{f\_y = sp.diff(f, y).subs(\{x:x\_0, y:y\_0\})}
\NormalTok{display(z\_0 + f\_x*(x{-}x\_0) + f\_y*(y{-}y\_0))}
\end{Highlighting}
\end{Shaded}

\subsection{Chain Rule (01/18)}\label{chain-rule-0118}

\subsubsection{\texorpdfstring{Warmup: Find the linearization of
\(f(x, y) = x^3y^4\) at the point \$(1,
1)}{Warmup: Find the linearization of f(x, y) = x\^{}3y\^{}4 at the point \$(1, 1)}}\label{warmup-find-the-linearization-of-fx-y-x3y4-at-the-point-1-1}

\begin{Shaded}
\begin{Highlighting}[numbers=left,,]
\NormalTok{f = x**3*y**4}
\NormalTok{x\_0 = 1}
\NormalTok{y\_0 = 1}
\NormalTok{z\_0 = f.subs(\{x:x\_0, y:y\_0\})}
\NormalTok{f\_x = sp.diff(f, x).subs(\{x:x\_0, y:y\_0\})}
\NormalTok{f\_y = sp.diff(f, y).subs(\{x:x\_0, y:y\_0\})}
\NormalTok{display(z\_0 + f\_x*(x{-}x\_0) + f\_y*(y{-}y\_0))}
\end{Highlighting}
\end{Shaded}

\subsubsection{Chain Rule (1-Variable)}\label{chain-rule-1-variable}

The ``normal'' chain rule formula:

\[
\frac{d}{dt}f(g(t))=f'(g(t))\cdot g'(t)
 = \frac{df}{dt}\cdot \frac{dg}{dt} = \frac{df}{dt}
\]

\subsubsection{Chain Rule (Multi
Variable)}\label{chain-rule-multi-variable}

Let \(w = f(x, y)\) where \(f\) is a differentiable function of \(x\)
and \(y\). If \(x = g(t)\) and \(y = h(t)\) where \(g\) and \(h\) are
differentiable functions of \(t\), then \(w\) is also a differentiable
function of \(t\) and
\(\frac{\mathrm{d}w}{\mathrm{d}t} = \frac{\partial w}{\partial x}\cdot \frac{\mathrm{d}x}{\mathrm{d}t} + \frac{\partial w}{\partial y}\cdot \frac{\mathrm{d}y}{\mathrm{d}t}\)

\subsubsection{Examples}\label{examples-8}

\paragraph{\texorpdfstring{If \(w=  x^2y-y^2\) where \(x\sin t\) and
\(y=e^t\), find \(\frac{\mathrm{d}w}{\mathrm{d}t}\) at
\(t=0\)}{If w=  x\^{}2y-y\^{}2 where x\textbackslash sin t and y=e\^{}t, find \textbackslash frac\{\textbackslash mathrm\{d\}w\}\{\textbackslash mathrm\{d\}t\} at t=0}}\label{if-w-x2y-y2-where-xsin-t-and-yet-find-fracmathrmdwmathrmdt-at-t0}

\begin{Shaded}
\begin{Highlighting}[numbers=left,,]
\NormalTok{w = x**2*y {-} y**2}
\NormalTok{w\_t = w.subs(\{x: sp.sin(t), y: sp.exp(t)\})}
\NormalTok{dw\_dt = sp.diff(w\_t, t)}
\NormalTok{display(dw\_dt.subs(t, 0))}
\end{Highlighting}
\end{Shaded}

\paragraph{\texorpdfstring{If \(w = xy + xz + yz\) and
\(x = t-1, y=t^2-1, z=t\), find
\(\frac{\mathrm{d}w}{\mathrm{d}t}\)}{If w = xy + xz + yz and x = t-1, y=t\^{}2-1, z=t, find \textbackslash frac\{\textbackslash mathrm\{d\}w\}\{\textbackslash mathrm\{d\}t\}}}\label{if-w-xy-xz-yz-and-x-t-1-yt2-1-zt-find-fracmathrmdwmathrmdt}

\begin{Shaded}
\begin{Highlighting}[numbers=left,,]
\NormalTok{w = x*y + x*z + y*z}
\NormalTok{w\_t = w.subs(\{x: t{-}1, y: t**2{-}1, z: t\})}
\NormalTok{display(sp.diff(w\_t, t))}
\end{Highlighting}
\end{Shaded}

\subsubsection{Homework}\label{homework-2}

\paragraph{\texorpdfstring{\(w = x^2+y^2, x=\cos t, y=\sin t\) at
\(t=\pi\)}{w = x\^{}2+y\^{}2, x=\textbackslash cos t, y=\textbackslash sin t at t=\textbackslash pi}}\label{w-x2y2-xcos-t-ysin-t-at-tpi}

\begin{Shaded}
\begin{Highlighting}[numbers=left,,]
\NormalTok{w = x**2 + y**2}
\NormalTok{w\_t = w.subs(\{x: sp.cos(t), y: sp.sin(t)\})}
\NormalTok{display(sp.diff(w\_t, t))}
\NormalTok{display(sp.diff(w\_t, t).subs(t, sp.pi))}
\end{Highlighting}
\end{Shaded}

\paragraph{\texorpdfstring{\(w = 2ye^x - \ln z, x = \ln(t^2 + 1), y = \arctan t, z=e^t\)
at
\(t=1\)}{w = 2ye\^{}x - \textbackslash ln z, x = \textbackslash ln(t\^{}2 + 1), y = \textbackslash arctan t, z=e\^{}t at t=1}}\label{w-2yex---ln-z-x-lnt2-1-y-arctan-t-zet-at-t1}

\begin{Shaded}
\begin{Highlighting}[numbers=left,,]
\NormalTok{w = 2*y*sp.exp(x) {-} sp.ln(z)}
\NormalTok{part\_wx = sp.diff(w, x)}
\NormalTok{part\_wy = sp.diff(w, y)}
\NormalTok{part\_wz = sp.diff(w, z)}
\NormalTok{dx\_dt = sp.diff(sp.ln(t**2 + 1), t)}
\NormalTok{dy\_dt = sp.diff(sp.atan(t), t)}
\NormalTok{dz\_dt = sp.diff(sp.exp(t), t)}
\NormalTok{dw\_dt = part\_wx * dx\_dt + part\_wy * dy\_dt + part\_wz * dz\_dt}
\NormalTok{display(dw\_dt)}
\NormalTok{display(dw\_dt.subs(t, 1))}
\end{Highlighting}
\end{Shaded}

\paragraph{\texorpdfstring{\(z = 4e^x\ln y, x=\ln (r\cos\theta), y=r\sin\theta; (r, \theta) = \left(2, \frac{\pi}{4}\right)\)}{z = 4e\^{}x\textbackslash ln y, x=\textbackslash ln (r\textbackslash cos\textbackslash theta), y=r\textbackslash sin\textbackslash theta; (r, \textbackslash theta) = \textbackslash left(2, \textbackslash frac\{\textbackslash pi\}\{4\}\textbackslash right)}}\label{z-4exln-y-xln-rcostheta-yrsintheta-r-theta-left2-fracpi4right}

\begin{Shaded}
\begin{Highlighting}[numbers=left,,]
\NormalTok{z = 4*sp.exp(x)*sp.ln(y)}
\NormalTok{part\_zx = sp.diff(z, x)}
\NormalTok{part\_zy = sp.diff(z, y)}
\NormalTok{dx\_dr = sp.diff(sp.ln(r*sp.cos(θ)), r)}
\NormalTok{dx\_dθ = sp.diff(sp.ln(r*sp.cos(θ)), θ)}
\NormalTok{dy\_dr = sp.diff(r*sp.sin(θ), r)}
\NormalTok{dy\_dθ = sp.diff(r*sp.sin(θ), θ)}
\NormalTok{dz\_dr = part\_zx * dx\_dr + part\_zy * dy\_dr}
\NormalTok{dz\_dθ = part\_zx * dx\_dθ + part\_zy * dy\_dθ}
\NormalTok{display(Latex(r"$\textbackslash{}frac\{\textbackslash{}partial z\}\{\textbackslash{}partial r\} = " + sp.latex(dz\_dr) + "$"))}
\NormalTok{display(Latex(r"$\textbackslash{}frac\{\textbackslash{}partial z\}\{\textbackslash{}partial\textbackslash{}theta\} = " + sp.latex(dz\_dθ) + "$"))}
\NormalTok{display(dz\_dr.subs(\{r:2, θ:sp.pi/4\}))}
\NormalTok{display(dz\_dθ.subs(\{r:2, θ:sp.pi/4\}))}
\end{Highlighting}
\end{Shaded}

\paragraph{\texorpdfstring{\(u = \frac{p-q}{q-r}, p = x+y+z, q = x - y +z, r = x+y-z; (x, y, z) = (\sqrt{3}, 2, 1)\)}{u = \textbackslash frac\{p-q\}\{q-r\}, p = x+y+z, q = x - y +z, r = x+y-z; (x, y, z) = (\textbackslash sqrt\{3\}, 2, 1)}}\label{u-fracp-qq-r-p-xyz-q-x---y-z-r-xy-z-x-y-z-sqrt3-2-1}

\begin{Shaded}
\begin{Highlighting}[numbers=left,,]
\NormalTok{u = (p{-}q)/(q{-}r)}
\NormalTok{part\_ux = sp.diff(u.subs(\{p: x+y+z, q: x{-}y+z, r: x+y{-}z\}), x)}
\NormalTok{part\_uy = sp.diff(u.subs(\{p: x+y+z, q: x{-}y+z, r: x+y{-}z\}), y)}
\NormalTok{display(Latex(r"$\textbackslash{}frac\{\textbackslash{}partial u\}\{\textbackslash{}partial x\} = " + sp.latex(part\_ux) + "$"))}
\NormalTok{display(Latex(r"$\textbackslash{}frac\{\textbackslash{}partial u\}\{\textbackslash{}partial y\} = " + sp.latex(part\_uy) + "$"))}
\NormalTok{display(Latex(r"$\textbackslash{}frac\{\textbackslash{}partial u\}\{\textbackslash{}partial z\} = \textbackslash{}frac\{4y\}\{({-}2y+2z)\^{}2\}$"))}
\NormalTok{display(part\_ux.subs(\{x:sp.sqrt(3), y:2, z:1\}))}
\NormalTok{display(part\_uy.subs(\{x:sp.sqrt(3), y:2, z:1\}))}
\NormalTok{display(Latex(r"${-}2$"))}
\end{Highlighting}
\end{Shaded}

\subsection{Related Rates and Implicit Differentiation
(01/22)}\label{related-rates-and-implicit-differentiation-0122}

\subsubsection{Related rates in Calc I}\label{related-rates-in-calc-i}

\begin{enumerate}
\def\labelenumi{\arabic{enumi})}
\tightlist
\item
  Come up with a formula (usually volume or something)
\item
  Pray to the math gods that you can somehow put the formula in terms of
  one variable
\item
  Take the derivatives, plug in tons of values and rates, and solve for
  the missing one
\end{enumerate}

In Calc III, you can remove step 2 and just use chain rule for step 3.

\subsubsection{Examples}\label{examples-9}

\paragraph{\texorpdfstring{A right circular cylinder with a n open top
has height \(h\), radius \(r\), and surface area \(A\). If
\(\frac{\mathrm{d}h}{\mathrm{d}t} = 3\) and
\(\frac{\mathrm{d}r}{\mathrm{d}t} = -2\), find
\(\frac{\mathrm{d}A}{\mathrm{d}t}\) when \(h=10\) and
\(r=5\).}{A right circular cylinder with a n open top has height h, radius r, and surface area A. If \textbackslash frac\{\textbackslash mathrm\{d\}h\}\{\textbackslash mathrm\{d\}t\} = 3 and \textbackslash frac\{\textbackslash mathrm\{d\}r\}\{\textbackslash mathrm\{d\}t\} = -2, find \textbackslash frac\{\textbackslash mathrm\{d\}A\}\{\textbackslash mathrm\{d\}t\} when h=10 and r=5.}}\label{a-right-circular-cylinder-with-a-n-open-top-has-height-h-radius-r-and-surface-area-a.-if-fracmathrmdhmathrmdt-3-and-fracmathrmdrmathrmdt--2-find-fracmathrmdamathrmdt-when-h10-and-r5.}

\begin{Shaded}
\begin{Highlighting}[numbers=left,,]
\NormalTok{h, r, t = sp.symbols(\textquotesingle{}h r t\textquotesingle{})}
\NormalTok{A = sp.pi*r**2 + 2*sp.pi*r*h}
\NormalTok{dhdt = 3}
\NormalTok{drdt = {-}2}
\NormalTok{dAdt = sp.diff(A, r)*drdt + sp.diff(A, h)*dhdt}
\NormalTok{display(dAdt.subs(\{h:10, r:5\}))}
\end{Highlighting}
\end{Shaded}

\subsubsection{Implicit Differentiation From Calc
I}\label{implicit-differentiation-from-calc-i}

\paragraph{\texorpdfstring{Find \(\frac{\mathrm{d}y}{\mathrm{d}x}\)
given that
\(y^3 + 4x^2 - 2xy +3x =  19\)}{Find \textbackslash frac\{\textbackslash mathrm\{d\}y\}\{\textbackslash mathrm\{d\}x\} given that y\^{}3 + 4x\^{}2 - 2xy +3x =  19}}\label{find-fracmathrmdymathrmdx-given-that-y3-4x2---2xy-3x-19}

\(y' = \frac{8x+2y-3}{3y^2-2x}\)

\paragraph{\texorpdfstring{Now suppose that
\(f(x,y) = y^3 + 4x^2 - 2xy +3x - 19\). Find
\(\frac{\partial f}{\partial x}\) and
\(\frac{\partial f}{\partial y}\).}{Now suppose that f(x,y) = y\^{}3 + 4x\^{}2 - 2xy +3x - 19. Find \textbackslash frac\{\textbackslash partial f\}\{\textbackslash partial x\} and \textbackslash frac\{\textbackslash partial f\}\{\textbackslash partial y\}.}}\label{now-suppose-that-fxy-y3-4x2---2xy-3x---19.-find-fracpartial-fpartial-x-and-fracpartial-fpartial-y.}

\begin{Shaded}
\begin{Highlighting}[numbers=left,,]
\NormalTok{display(sp.diff(f, x))}
\NormalTok{display(sp.diff(f, y))}
\end{Highlighting}
\end{Shaded}

If \(f(x,y) = 0\) defines \(y\) implicitly as a differentiable fnuction
of \(x\), then
\(\frac{\mathrm{d}y}{\mathrm{d}x} = -\frac{\frac{\partial f}{\partial x}}{\frac{\partial f}{\partial y}}\)

Similarly, if \(f(x, y, z) = 0\) defines \(z\) implicitly as a
differentiable function of \(x\) and \(y\), then

\[
\frac{\mathrm{d}z}{\mathrm{d}x} = -\frac{\frac{\partial f}{\partial x}}{\frac{\partial f}{\partial z}} \text{ and } \frac{\mathrm{d}z}{\mathrm{d}y} = -\frac{\frac{\partial f}{\partial y}}{\frac{\partial f}{\partial z}}
\]

\subsection{Directional Derivatives and Gradient Vectors
(01/25)}\label{directional-derivatives-and-gradient-vectors-0125}

Suppose that \(f\) is differentiable at \((x_0, y_0)\) and
\(\vec{u} = \langle u_1, u_2 \rangle\) is any \ul{unit} vector. Then the
directional derivative of \(f\) at \((x_0, y_0)\) in the direction of
\(\vec{u}\) is given by:
\(D_{\vec{u}}f(x_0, y_0) = f_x(x_0, y_0)u_1 + f_y(x_0, y_0)u_2\)

In other words, this is partial x times the x component of the unit
vector plus partial y times the y component of the unit vector.

The directional derivative is also denoted by:

\((D_uf)_{p_0}\); ``The derivative of \(f\) at \(P_0\) in the direction
of \(u\)''

\subsubsection{\texorpdfstring{Find the directional derivative of
\(f(x,y) = x^2\sin 2y\) at \(\left(1, \frac{\pi}{2}\right)\) in the
direction of
\(\vec{v} = 3\vec{i} - 4\vec{j}\).}{Find the directional derivative of f(x,y) = x\^{}2\textbackslash sin 2y at \textbackslash left(1, \textbackslash frac\{\textbackslash pi\}\{2\}\textbackslash right) in the direction of \textbackslash vec\{v\} = 3\textbackslash vec\{i\} - 4\textbackslash vec\{j\}.}}\label{find-the-directional-derivative-of-fxy-x2sin-2y-at-left1-fracpi2right-in-the-direction-of-vecv-3veci---4vecj.}

\begin{align*}
f_x &= 2x\sin 2y \\
f_y &= 2x^2\cos 2y \\
f_x(1, \frac{\pi}{2}) &= 0 \\
f_y(1, \frac{\pi}{2}) &= -2 \\
\vec{u_1} &= \frac{3}{5} \\
\vec{u_2} &= -\frac{4}{5} \\ 
D_{\vec{u}}f(1, \frac{\pi}{2}) &= 0 \cdot \frac{3}{5} - 2 \cdot -\frac{4}{5} \\
&= \frac{8}{5}
\end{align*}

\subsubsection{Thinking questions}\label{thinking-questions}

Not too bad, right?

What if I wanted the directional derivative of \(f(x, y)\) in the
direction of \(v = i\)? What can you tell me about that derivative?

\(f_x\)

What if I just wanted it in the direction of \$v = j \$ ?

\(f_u\)

So, all of this time, when you've been findign \(f_x\) and \(f_y\) at a
point \((x_0, y_0)\), you've really just been finding the components to
a (not unit) vector that contains both derivatives

\subsubsection{The Gradient}\label{the-gradient}

Let \(z = f(x, y)\) be a function of \(x\) and \(y\) such that \(f_x\)
and \(f_y\) exist. Then the gradient of \(f\) denoted by
\(\nabla f(x, y)\) is the vector:

\(\nabla f(x, y) = f_x(x, y)\vec{i} + f_y(x, y)\vec{j}\)

\subsubsection{\texorpdfstring{Find the gradient of
\(f(x, y) = y\ln x + xy^2\) at the point
\((e^3, 2)\)}{Find the gradient of f(x, y) = y\textbackslash ln x + xy\^{}2 at the point (e\^{}3, 2)}}\label{find-the-gradient-of-fx-y-yln-x-xy2-at-the-point-e3-2}

\begin{align*}
f_x &= \frac{y}{x} + y^2 \\
f_y &= \ln x + 2xy \\
\nabla f(x, y) &= \langle \frac{2}{e^3} + 4 , 3+4e^3 \rangle
\end{align*}

Wait a second\ldots{} go back to how we calculate a directional
derivative\ldots{}

\(D_{\vec{u}}f(x_0, y_0) = f_x(x_0, y_0)u_1 + f_y(x_0, y_0)u_2\)

This is the same as the dot product of the gradient and the unit vector!

\(D_{\vec{u}}f(x_0, y_0) = \nabla f(x_0, y_0) \cdot \vec{u}\)

\subsubsection{Homework}\label{homework-3}

Find the gradient of the following:

\paragraph{\texorpdfstring{\(g(x, y) = y-x^2, (-1, 0)\)}{g(x, y) = y-x\^{}2, (-1, 0)}}\label{gx-y-y-x2--1-0}

\begin{Shaded}
\begin{Highlighting}[numbers=left,,]
\NormalTok{g = y {-} x**2}
\NormalTok{g\_x = sp.diff(g, x)}
\NormalTok{g\_y = sp.diff(g, y)}
\NormalTok{display(g\_x.subs(\{x:{-}1, y:0\}))}
\NormalTok{display(g\_y.subs(\{x:{-}1, y:0\}))}
\end{Highlighting}
\end{Shaded}

Find the directional derivative of \(f\) at \(P_0\) in the direction of
\(A\).

\paragraph{\texorpdfstring{\(f(x, y) = 2xy - 3y^2, P_0 = (5, 5), A = \langle 4, 3 \rangle\)}{f(x, y) = 2xy - 3y\^{}2, P\_0 = (5, 5), A = \textbackslash langle 4, 3 \textbackslash rangle}}\label{fx-y-2xy---3y2-p_0-5-5-a-langle-4-3-rangle}

\begin{Shaded}
\begin{Highlighting}[numbers=left,,]
\NormalTok{f = 2*x*y {-} 3*y**2}
\NormalTok{f\_x = sp.diff(f, x).subs(\{x:5, y:5\})}
\NormalTok{f\_y = sp.diff(f, y).subs(\{x:5, y:5\})}
\NormalTok{A = sp.sqrt(4**2 + 3**2)}
\NormalTok{u\_1 = 4/A}
\NormalTok{u\_2 = 3/A}
\NormalTok{display(f\_x*u\_1 + f\_y*u\_2)}
\end{Highlighting}
\end{Shaded}

\paragraph{\texorpdfstring{\(f(x, y) = x- \left(\frac{y^2}{x}\right) + \sqrt{3}\sec^{-1}(2xy), P_0 = (1, 1), A = \langle 12, 5 \rangle\)}{f(x, y) = x- \textbackslash left(\textbackslash frac\{y\^{}2\}\{x\}\textbackslash right) + \textbackslash sqrt\{3\}\textbackslash sec\^{}\{-1\}(2xy), P\_0 = (1, 1), A = \textbackslash langle 12, 5 \textbackslash rangle}}\label{fx-y-x--leftfracy2xright-sqrt3sec-12xy-p_0-1-1-a-langle-12-5-rangle}

\begin{Shaded}
\begin{Highlighting}[numbers=left,,]
\NormalTok{f = x {-} y**2/x + sp.sqrt(3)*sp.asec(2*x*y)}
\NormalTok{f\_x = sp.diff(f, x).subs(\{x:1, y:1\})}
\NormalTok{f\_y = sp.diff(f, y).subs(\{x:1, y:1\})}
\NormalTok{A = sp.sqrt(12**2 + 5**2)}
\NormalTok{u\_1 = 12/A}
\NormalTok{u\_2 = 5/A}
\NormalTok{display(f\_x*u\_1 + f\_y*u\_2)}
\end{Highlighting}
\end{Shaded}

\paragraph{\texorpdfstring{\(f(x, y) = xy + yz + zx, P_0 = (1, -1, 2), A = \angle 3, 6, -2 \rangle\)}{f(x, y) = xy + yz + zx, P\_0 = (1, -1, 2), A = \textbackslash angle 3, 6, -2 \textbackslash rangle}}\label{fx-y-xy-yz-zx-p_0-1--1-2-a-angle-3-6--2-rangle}

\begin{Shaded}
\begin{Highlighting}[numbers=left,,]
\NormalTok{x, y, z = sp.symbols(\textquotesingle{}x y z\textquotesingle{}, real=True)}
\NormalTok{f = x*y + y*z + z*x}
\NormalTok{f\_x = sp.diff(f, x).subs(\{x:1, y:{-}1, z:2\})}
\NormalTok{f\_y = sp.diff(f, y).subs(\{x:1, y:{-}1, z:2\})}
\NormalTok{f\_z = sp.diff(f, z).subs(\{x:1, y:{-}1, z:2\})}
\NormalTok{A = sp.sqrt(3**2 + 6**2 + ({-}2)**2)}
\NormalTok{u\_1 = 3/A}
\NormalTok{u\_2 = 6/A}
\NormalTok{u\_3 = {-}2/A}
\NormalTok{display(f\_x*u\_1 + f\_y*u\_2 + f\_z*u\_3)}
\end{Highlighting}
\end{Shaded}

\subsection{More Gradient and Tangent Planes
(01/29)}\label{more-gradient-and-tangent-planes-0129}

\subsubsection{Properties of Gradients}\label{properties-of-gradients}

\(D_{\vec{u}}f = \nabla f \cdot \vec{u} = ||\nabla f||||u||\cos\theta\)

In order to maximize this value, \(\cos\theta\) must be 1.

Some consequences:

\begin{itemize}
\tightlist
\item
  \(f\) increases most rapidly when \(\cos\theta = 1 (\theta = 0)\),
  which implies that \(\nabla f\) and \(\vec{u}\) are in the same
  direction.
\item
  This maximum rate of increase \(= ||\nabla f||\)
\item
  \(f\) decreases most rapidly when \(\cos\theta = -1 (\theta = \pi)\),
  which implies that \(\nabla f\) and \(\vec{u}\) are in opposite
  directions.
\item
  This maximum rate of decrease \(= -||\nabla f||\)
\end{itemize}

Bit more interesting: \ul{What if I wanted the directional derivative to
be zero?}

\(D_{\vec{u}}f = \nabla f \cdot \vec{u} = 0\)

When the dot product equals 0.

\subsubsection{Examples}\label{examples-10}

\paragraph{\texorpdfstring{Find the maximum rate of increase/decrease of
the function \(f(x, y) = x^2+y^2\) at
\((1, 3)\)}{Find the maximum rate of increase/decrease of the function f(x, y) = x\^{}2+y\^{}2 at (1, 3)}}\label{find-the-maximum-rate-of-increasedecrease-of-the-function-fx-y-x2y2-at-1-3}

\begin{Shaded}
\begin{Highlighting}[numbers=left,,]
\NormalTok{f = x**2 + y**2}
\NormalTok{f\_x = sp.diff(f, x)}
\NormalTok{f\_y = sp.diff(f, y)}
\NormalTok{display(Latex(r"$\textbackslash{}nabla f(x, y) = " + sp.latex(sp.Matrix([f\_x, f\_y])) + "$"))}
\NormalTok{display(Latex(r"$||\textbackslash{}nabla f(x, y)|| = " + sp.latex(sp.sqrt(f\_x**2 + f\_y**2)) + "$"))}
\NormalTok{display(Latex(r"$||\textbackslash{}nabla f(1, 3)|| = " + sp.latex(sp.sqrt(f\_x.subs(\{x:1, y:3\})**2 + f\_y.subs(\{x:1, y:3\})**2)) + "$"))}
\end{Highlighting}
\end{Shaded}

\paragraph{\texorpdfstring{The temperature in Celsius on the surface of
a metal plate is \(T(x, y) = 20-4x^2-y^2\) where \(x\) and \(y\) are
measured in cm. In what direction from \((2, -3)\) does the
temperature}{The temperature in Celsius on the surface of a metal plate is T(x, y) = 20-4x\^{}2-y\^{}2 where x and y are measured in cm. In what direction from (2, -3) does the temperature}}\label{the-temperature-in-celsius-on-the-surface-of-a-metal-plate-is-tx-y-20-4x2-y2-where-x-and-y-are-measured-in-cm.-in-what-direction-from-2--3-does-the-temperature}

\begin{enumerate}
\def\labelenumi{\alph{enumi}.}
\tightlist
\item
  Increase most rapidly?
\end{enumerate}

\begin{Shaded}
\begin{Highlighting}[numbers=left,,]
\NormalTok{T = 20 {-} 4*x**2 {-} y**2}
\NormalTok{T\_x = sp.diff(T, x)}
\NormalTok{T\_y = sp.diff(T, y)}
\NormalTok{\# The gradient at (2, {-}3)}
\NormalTok{gradient = sp.Matrix([T\_x.subs(\{x:2, y:{-}3\}), T\_y.subs(\{x:2, y:{-}3\})])}
\NormalTok{magnitude = sp.sqrt(T\_x.subs(\{x:2, y:{-}3\})**2 + T\_y.subs(\{x:2, y:{-}3\})**2)}
\NormalTok{unit\_vector = gradient/magnitude}
\NormalTok{display(unit\_vector)}
\end{Highlighting}
\end{Shaded}

\begin{enumerate}
\def\labelenumi{\alph{enumi}.}
\setcounter{enumi}{1}
\tightlist
\item
  Decrease most rapidly?
\end{enumerate}

\begin{Shaded}
\begin{Highlighting}[numbers=left,,]
\NormalTok{display({-}unit\_vector)}
\end{Highlighting}
\end{Shaded}

\begin{enumerate}
\def\labelenumi{\alph{enumi}.}
\setcounter{enumi}{2}
\tightlist
\item
  Not change at all?
\end{enumerate}

\begin{Shaded}
\begin{Highlighting}[numbers=left,,]
\NormalTok{display(sp.Matrix([6, 16]), sp.Matrix([{-}6, {-}16]))}
\end{Highlighting}
\end{Shaded}

\subsubsection{More Applications of
Gradients}\label{more-applications-of-gradients}

Given a surface \(z=f(x, y), [f(x, y)-z=0]\), the plane tangent to \(z\)
at the point \(P_0(x_0, y_0, z_0)\) is given by:

The equation of a plane is given by:

\(f_x(x_0, y_0, z_0)(x-x_0)+f_y(x_0, y_0, z_0)(y-y_0)+f_z(x_0, y_0, z_0)(z-z_0) = 0\)
where \(f_x, f_x, f_z \in \nabla (f(x, y)-z)\)

The equations of the normal line to the surface are given by:

\begin{itemize}
\tightlist
\item
  \(x = x_0 + f_x(x_0, y_0, z_0)t\)
\item
  \(y = y_0 + f_y(x_0, y_0, z_0)t\)
\item
  \(z = z_0 + f_z(x_0, y_0, z_0)t\)
\end{itemize}

\newpage{}

\subsection{Local Extrema and Saddle Points
(02/02)}\label{local-extrema-and-saddle-points-0202}

In multivariable calculus, mins/maxes on a closed region occur in the
following conditions:

\begin{itemize}
\tightlist
\item
  An interior point where \textbf{both} first partial derivatives equal
  0 (Critical point!)
\item
  An interior point where one \textbf{or} both first partial derivatives
  DNE (Critical point?)
\item
  Any boundary point of the region (End point)
\end{itemize}

What would a tangent plane look like at one of those interior critical
points?

\(0(x...) + 0(y...)+f_z(z...) = 0\)

\subsubsection{Types of Extrema}\label{types-of-extrema}

\begin{itemize}
\tightlist
\item
  \(f(a, b)\) is a relative max of \(f\) if
  \(f(a, b) \ge f(x, y) \forall\) domain points in an open disk centered
  at \((a, b)\)
\item
  \(f(a, b)\) is a relative min of \(f\) if
  \(f(a, b) \le f(x, y) \forall\) domain points in an open disk centered
  at \((a, b)\)
\item
  If \((a, b)\) is a critical point \((f_x = f_y = 0)\) and there are
  domain points where \(f(x, y) > f(a,b)\) and where
  \(f(x,y) < f(a,b)\), then \((a, b, f(a, b))\) is a saddle point.
\end{itemize}

\subsubsection{The Second Partials Test}\label{the-second-partials-test}

Suppose \(f(x,y)\) and its first and second partial derivatives are
continuous on an open region centered at \((a, b)\) and
\(f_x(a, b) = f_y(a, b) = 0\). Then:

\begin{enumerate}
\def\labelenumi{\alph{enumi})}
\tightlist
\item
  \(f\) has a rel \ul{max} at \((a, b)\) if \(f_{xx} < 0\) AND
  \(f_{xx}f_{yy}-(f_{xy})^2 > 0\) at \((a, b)\)
\end{enumerate}

\begin{itemize}
\tightlist
\item
  (if \(f\) is concave down in the x and y direction, then it's a
  relative max)
\end{itemize}

\begin{enumerate}
\def\labelenumi{\alph{enumi})}
\setcounter{enumi}{1}
\tightlist
\item
  \(f\) has a rel \ul{min} at \((a, b)\) if \(f_{xx} > 0\) AND
  \(f_{xx}f_{yy}-(f_{xy})^2 > 0\) at \((a, b)\)
\end{enumerate}

\begin{itemize}
\tightlist
\item
  (if \(f\) is concave up in the x and y direction, then it's a relative
  min)
\end{itemize}

\begin{enumerate}
\def\labelenumi{\alph{enumi})}
\setcounter{enumi}{2}
\tightlist
\item
  \(f\) has a saddle point if \(f_{xx}f_{yy} - (f_{xy})^2 < 0\) at
  \((a, b)\)
\end{enumerate}

\begin{itemize}
\tightlist
\item
  (if the concavity of \(f\) disagree \textbf{or} \(f_{xy}\) is too
  large, then it's a saddle point)
\end{itemize}

\begin{enumerate}
\def\labelenumi{\alph{enumi})}
\setcounter{enumi}{3}
\tightlist
\item
  The test is inconclusive if \(f_{xx}f_{yy} - (f_{xy})^2 = 0\) at
  \((a, b)\)
\end{enumerate}

\newpage{}

\subsection{Absolute Extrema (02/07)}\label{absolute-extrema-0207}

Recap of the extreme value theorem (EVT):\\
If \(f(x)\) is continuous over \([a, b]\), then \(f(x)\) \emph{must}
have an absolute max \emph{and} min.

\subsubsection{Calc I Approach}\label{calc-i-approach}

\begin{enumerate}
\def\labelenumi{\arabic{enumi})}
\tightlist
\item
  Create list of candidates (critical points, end points)
\item
  Brute force which one is largest/smallest. Plug and chug
\end{enumerate}

\subsubsection{Problems with Calc I Approach in
3D}\label{problems-with-calc-i-approach-in-3d}

It's not that different in multi! Although\ldots{} we don't have
endpoints because we don't have \emph{ends}. In 2D, boundaries to
possible values of \(x\) are 1-dimensional ({[}little \(x\), big
\(x\){]}).

In 3D, both \textbf{x} and \textbf{y} can have boundaries, so we
\emph{could} have {[}little \(x\), big \(x\){]} and {[}little \(y\), big
\(y\){]}. But that boundary system describes a rectangle\ldots{}

What if I wanted boundaries that looked like a triangle? A circle?
Anything else\ldots?

\subsubsection{Multi Approach}\label{multi-approach}

To find absolute extrema of a continuous function \(f(x,y)\) on a closed
bounded region \(R\):\\
1) Graph region \(R\) 2) Find the critical points of \(f\) 3) Find the
boundary points of \(R\) which are also critical points 4) Find the
coordinates of corners of \(R\) 5) Evaluate \(f\) \emph{at all these
points} and answer the question

\newpage{}

\subsection{Lagrange Multipliers
(02/08)}\label{lagrange-multipliers-0208}

When looking for a max/min on a surface constrained by another function,
the two gradients of those functions at the maximum/minimum point are
multiples of each other. That multiplier is the \textbf{Lagrange
Multiplier}.

\subsubsection{More formal definition}\label{more-formal-definition}

Suppose \(f(x,y,z)\) and \(g(x,y,z)\) are functions with continuous
first partial derivatives and \(\nabla g(x,y,z) \ne 0\) on the surface
\(g(x,y,z) = 0\). Suppose also that the minimum/maximum of
\(f(x, y, z)\) subject to the constraint \(g(x,y,z) = 0\) occurs at
\((x_0, y_0, z_0)\). Then
\(\nabla f(x_0, y_0, z_0) = \lambda \nabla g(x_0, y_0, z_0)\) for some
non-zero constant \(\lambda\) called the Lagrange multiplier.

Essentially, it's saying the two functions have a proportional set of
partial derivatives.

\subsubsection{Steps}\label{steps}

\paragraph{Two variables}\label{two-variables}

\begin{enumerate}
\def\labelenumi{\arabic{enumi})}
\tightlist
\item
  Set up the equation \(\nabla f = \lambda \nabla g\)
\item
  Solve each equation for \(\lambda\) and set them equal to create an
  equation with just \(x\) and \(y\)
\item
  Solve for a variable and plug it into the constraint equation
\item
  Solve the constraint equation and back substitute to find the other
  variable
\item
  Evaluate the function at that point to find the max/min
\end{enumerate}

\paragraph{Three variables}\label{three-variables}

\begin{enumerate}
\def\labelenumi{\arabic{enumi})}
\tightlist
\item
  Set up the equation \(\nabla f = \lambda \nabla g\)
\item
  Solve each equation for a non-\(\lambda\) variable so that each is a
  function of \(\lambda\)
\item
  Plug all lambda equations into constraint to create equations in terms
  of one variable
\item
  Solve for lambda and back substitute into each variable
\item
  If you have multiple points, evaluate the function at each point to
  find the max/min
\end{enumerate}

\subsubsection{More than one multiplier}\label{more-than-one-multiplier}

When optimizing with two constraints, \(g\) and \(h\), we introduce a
second multiplier, \(\mu\) and our system becomes: ~ -
\(f_x = \lambda g_x + \mu h_x\) - \(f_y = \lambda g_y + \mu h_y\) -
\(f_z = \lambda g_z + \mu h_z\)



\end{document}
